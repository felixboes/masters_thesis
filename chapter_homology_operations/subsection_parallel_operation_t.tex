\subsection{The Operation \texorpdfstring{$T$}{T} via the Dual Ehrenfried Complex}
In this subsection, we construct the operation $T$ in terms of the dual Ehrenfried complex.
Geometrically speaking, we start with a combinatorial cell $\Sigma$ and have to introduce two new slits,
one rotating in a puncture and the other on top of all other slits.
In the algebraic model, there is no notion of rotating slits but it is easy to come up with the right definition.
In order to reduce the cohomological degree of the resulting cell by one, we append every cell by a transposition $(\ul{p_r+1}_r\ c)$ with $c$ a symbol in a puncture.
Using the geometric intuition of jumping slits and relevant $\kappa^\ast$-sequences, it is easy to see that we defined a coboundary map.

\begin{defi}
    \label{homology_operations:parallel_T:symbols_of_a_puncture}
    \index{symbols of a puncture}
    \symbolindex[p]{$\punc(\Sigma)$}{The symbols of the $m$ cycles of $\Sigma$ corresponding to the punctures.}{Definition \ref{homology_operations:parallel_T:symbols_of_a_puncture}}
    Let $\Sigma = \inhom = \homog$ be a top dimensional, non-degenerate cell in $P(h,m; r_1, \ldots, r_n)$.
    Denote the cycles of $\sigma_h$ that correspond to the $m$ punctures by $\alpha_1, \ldots, \alpha_m$.
    The {\bfseries symbols corresponding to the punctures of $\Sigma$} are
    \[
        \punc(\Sigma) = \supp(\alpha_1, \ldots, \alpha_m) \,.
    \]
\end{defi}
\begin{notation}
    Since $T$ adds a new slit above all other slits, we obtain the ordered partition $p+1 = p_1 + \ldots + p_{r-1} + (p_r+1)$.
    In particular, the largest symbol is $\ul{p_r+1}_r = p_r+1$.
\end{notation}

\begin{defi}
    \label{homology_operations:parallel_T:defn_on_cells}
    \symbolindex[t]{$T(\Sigma)$}{The homology operation $T$ of $\Sigma$ in terms of the dual Ehrenfried complex.}{Definition \ref{homology_operations:parallel_T:defn_on_cells}}
    Then the operation $T$ is defined on generators $\Sigma \in \E^\ast$ by
    \[
        T(\Sigma) = \sum_{c \in \punc(\Sigma)} \Sigma_c 
    \]
    where
    \[
        \Sigma_c = \big(\ (\ul{p_r+1}_r\ c) \mid \tau_q \mid \ldots \mid \tau_1 \big) \,.
    \]
\end{defi}

\begin{prop}
    \label{homology_operations:parallel_T:T_is_a_cochain_map}
    The operation $T$ defines a cochain map
    \begin{multline*}
        T \colon \E^{\ast+1}(h,m; r_1, \ldots, r_n) = \E^\ast(h,m; r_1, \ldots, r_n) \otimes \Z[1] \to\\
        \E^\ast(h+1,m-1;r_1, \ldots, r_n) \,.
    \end{multline*}
\end{prop}

\begin{lem}
    \label{homology_operations:parallel_T:T_is_well_defined}
    If $\Sigma$ is top dimensional, non-degenerate cell of bidegree $(p,h)$ in $P_{g,n}^m[(r_1, \ldots, r_n)]$, then
    every term $\Sigma_c$ of $T(\Sigma)$ is a top dimensional, non-degenerate cell of bidegree $(p+1,h+1)$ in $P(h+1,m-1; r_1, \ldots, r_n)$.
\end{lem}

\begin{proof}
    Consider $\Sigma = \inhom = \homog$ and let $\Sigma_c = (x_{h+1} \mid \ldots \mid x_1) = ( \sigma_c : \sigma_h : \ldots : \sigma_0)$ be a term of $T(\Sigma)$, i.e.\ 
    $\Sigma_c = \big( (\ul{p_r+1}_r\ c)\mid \tau_q \mid \ldots \mid \tau_1 \big)$ and $\sigma_c = (\ul{p_r+1}_r\ c)\sigma_h$, with $c \in \punc(\Sigma)$.
    
    The following is evident:
    $N(\Sigma_c) = N(\Sigma) + 1 = h+1$, $\Sigma_c$ is connected, the levels are ordered and there is neither $1 = x_i$ nor a common fixed point of $x_{q+1}, \ldots, x_1$.
    From $\sigma_c = (\ul{p_r+1}_r\ c) \sigma_h$ and $\ul{p_r+1}_r \not\in \punc(\Sigma) \ni c$ we deduce
    \[
        n(\Sigma_c) = n(\Sigma) \mspc{and}{20} m(\Sigma_c) = m(\Sigma) - 1\,.
    \]
\end{proof}

\begin{lem}
    \label{homology_operations:parallel_T:T_commutes_with_del_pi}
    We have
    \[
        T\del_\KK^\ast\pi^\ast = \del_\KK^\ast\pi^\ast T \,.
    \]
\end{lem}

In order to prove the lemma, we use Proposition \ref{cellular_models:dual_ehrenfried:cob_tr_equals_cob} to show that every term in $T\del_\KK^\ast\pi^\ast(\Sigma)$ occures in $\del_\KK^\ast\pi^\ast T(\Sigma)$.
Then, using Proposition \ref{cellular_models:dual_ehrenfried:cob_tr_equals_cob} again, the difference of both sums is zero.

\begin{proof}
    By Proposition \ref{cellular_models:dual_ehrenfried:cob_tr_equals_cob}, the terms of $\del_\KK^\ast \pi^\ast(\Sigma)$ correspond bijectively to all coboundary traces of $\Sigma$.
    Applying $T$, every term $x$ of $T\del_\KK^\ast \pi^\ast(\Sigma)$ is identified with an $i\Th$ coboundary trace $a = a(x)$ and a symbol $c = c(x)$ corresponding to one of the punctures of $a.\Sigma$.
    
    If $c \neq i$, we identify $x$ with the term $\tilde a .\Sigma_{\tilde c}$, where $\tilde c = d_i^\Delta(c)$ and $\tilde a$ is the $i\Th$ coboundary trace of $\Sigma_{\tilde c}$ with
    \begin{align}
        \label{homology_operations:parallel_T:T_commutes_with_del_pi:c_neq_i}
        \tilde a_j = \begin{cases} a_j & j \le h \\ (\ul{p_r+1}_r\ c)(a_h) & j = h+1 \end{cases}
    \end{align}
    as both $\tilde a \in T_i(\Sigma_{\tilde c})$ and $x = \tilde a.\Sigma_{\tilde c}$ are readily verified.
    
    Otherwise, i.e\ if $c = i$, we identify $x$ with the term $a'.\Sigma_{c'}$ where $c' = d_i^\Delta(a_h)$ and  $a'$ is the $i\Th$ coboundary trace of $\Sigma_{c'}$ with
    \begin{align}
        \label{homology_operations:parallel_T:T_commutes_with_del_pi:c_eq_i}
        a_j' = \begin{cases} a_j & j \le h \\ a_h & j = h+1 \end{cases}
    \end{align}
    as both $a' \in T_i(\Sigma_{c'})$ and $x = a'.\Sigma_{c'}$ are again readily verified.
    
    Observe that in case \eqref{homology_operations:parallel_T:T_commutes_with_del_pi:c_neq_i} we have
    \[
        \tilde a_{h+1} = S_i(\ul{p_r+1}_r\ \tilde c)(a_j)
    \]
    and in case \eqref{homology_operations:parallel_T:T_commutes_with_del_pi:c_eq_i} we have
    \[
        \tilde a_{h+1} \neq S_i(\ul{p_r+1}_r\ c')(a_j) \,.
    \]
    We identify the terms of $T\del_\KK^\ast\pi^\ast(\Sigma)$ with all coboundary traces $a = (a_{h+1} : \ldots : a_0)$ of all terms of $T(\Sigma)$ that satisfy both
    \[
        a_j \neq a_{j-1} \mspc{for some}{10} j \le h \mspc{and}{30} a_j \neq (S_i\tau_j)(a_{j-1}) \mspc{for some}{10} j \le h \,.
    \]
    The remaining terms of $\del_\KK^\ast\pi^\ast T(\Sigma) - T\del_\KK^\ast\pi^\ast(\Sigma)$
    are identified with the coboundary traces of all terms $\Sigma_c = (x_{h+1} \mid \ldots \mid x_1)$ of $T(\Sigma)$ that satisfy
    \begin{align}
        \label{homology_operations:parallel_T:T_commutes_with_del_pi:a_j_equal_a_j_1}
        a_j = a_{j-1} = i+1 \mspc{for all}{10} j \le h \mspc{and}{30} a_{h+1} \neq a_h
        \intertext{or}
        \label{homology_operations:parallel_T:T_commutes_with_del_pi:a_j_equal_S_i_x_j}
        a_j = (S_i x_j)(a_{j-1})  \mspc{for all}{10} j \le h \mspc{and}{30} a_{h+1} \neq (S_i x_{h+1})(a_h) \,.
    \end{align}
    Let us reformulate \eqref{homology_operations:parallel_T:T_commutes_with_del_pi:a_j_equal_a_j_1} and \eqref{homology_operations:parallel_T:T_commutes_with_del_pi:a_j_equal_S_i_x_j}.
    Clearly
    \begin{align}
        a \in T_i(\Sigma_c) \text{ satisfies \eqref{homology_operations:parallel_T:T_commutes_with_del_pi:a_j_equal_a_j_1}}
            &\iff S_i( \ul{p_r+1}_r\ c )(i+1) \neq i+1 \\
            &\iff c = i \\
            &\iff a = (\ul{p_r+1}_r: i+1 : \ldots : i+1) \mspc{and}{10} i \in \punc(\Sigma) \\
            \label{homology_operations:parallel_T:T_commutes_with_del_pi:a_j_equals_a_j_1_equiv}
            &\iff a = (\ul{p_r+1}_r: i+1 : \ldots : i+1) \mspc{and}{10} \sigma_h(i) \in \punc(\Sigma)
    \end{align}
    and
    \begin{align}
        a \in T_{i+1}(\Sigma_c) \text{ satisfies \eqref{homology_operations:parallel_T:T_commutes_with_del_pi:a_j_equal_S_i_x_j}}
            &\iff S_{i+1}( \ul{p_r+1}_r\ c )(a_h) \neq a_h
            \intertext{using $a_j = S_{i+1}(\tau_j\cdots\tau_1)(i+2) = s_{i+1}^\Delta(\sigma_j(i))$ yields}
            &\iff c = a_h = \sigma_h(i) \\
            &\iff a = (s_{i+1}^\Delta(\sigma_h(i)) : s_{i+1}^\Delta(\sigma_h(i)) : \ldots : s_{i+1}^\Delta(\sigma_0(i))) \notag \\
            \label{homology_operations:parallel_T:T_commutes_with_del_pi:a_j_equal_S_i_x_j_equiv}
            &\phantom{{}\iff} \mspc{and}{10} \sigma_h(i) \in \punc(\Sigma) \,.
    \end{align}
    By \eqref{homology_operations:parallel_T:T_commutes_with_del_pi:a_j_equals_a_j_1_equiv} and \eqref{homology_operations:parallel_T:T_commutes_with_del_pi:a_j_equal_S_i_x_j_equiv}
    the remaining terms of $\del_\KK^\ast\pi^\ast T(\Sigma) - T\del_\KK^\ast\pi^\ast(\Sigma)$ come in pairs, where
    \[
        a \in T_i(\Sigma_i) \mspc{is paired with}{30} a' \in T_{i+1}(\Sigma_{\sigma_h(i)})
    \]
    and a direct computation shows
    \[
        a.\Sigma_i = \big(\ (\ul{p_r+1}_r\ i+1) \mid S_{i+1}\tau_q \mid \ldots \mid S_{i+1}\tau_1 \big) = a'.\Sigma_{\sigma_h(i)}
    \]
    which finishes the proof:
    \[
        \del_\KK^\ast\pi^\ast T(\Sigma) - T\del_\KK^\ast\pi^\ast(\Sigma)
            = \sum_{a \in T_i(\Sigma_i)} (-1)^i a.\Sigma_i + (-1)^{i+1} a'.\Sigma_{\sigma_h(i)} = 0 \,.
    \]
\end{proof}

\begin{proof}[Proof of Proposition \ref{homology_operations:parallel_T:T_is_a_cochain_map}]
    The map $T$ is well defined by Lemma \ref{homology_operations:parallel_T:T_is_well_defined},
    it commutes up to $\kappa^\ast$ with $\del_\E^\ast = \kappa^\ast \del_\KK^\ast \pi^\ast$ by Lemma \ref{homology_operations:parallel_T:T_commutes_with_del_pi},
    so the comparision of relevant $\kappa^\ast$-sequences using Lemma \ref{cellular_models:dual_ehrenfried:f_dual_vanishes_at_monotonous_spot} ends the proof.
\end{proof}
