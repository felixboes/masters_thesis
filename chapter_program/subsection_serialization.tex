\subsection{Serialization}
\label{program:kappa:serialization}

The transformation of data and objects of a running computer program into storable information (which can be saved on a hard disk) is called serialization.
In our project, we want to generate basis elements and differentials to save them for later use, i.e.\ we want to serialize them.
Therefore, we provide the general purpose template functions
\begin{lstlisting}
template < class StorableType >
void save_to_file_bz2
    ( StorableType& t, std::string filename, bool print_duration=true );
\end{lstlisting}
and
\begin{lstlisting}
template < class StorableType >
void load_from_file_bz2
    ( StorableType& t, std::string filename, bool print_duration=true);
\end{lstlisting}
Both functions, use the \progname{bzip2} algorithm for fast file compresseion, so we produce much smaller files.

\subsubsection{Storable Types}
A type can be handled by the above template functions, if
it is defined by the \cppeleven\ standard (such as \progclass{int} or \progclass{std}::\progclass{vector}\progname{<} \progclass{char} \progname{>}) or the {\bf boost \cpp library}.
Otherwise your class has to meet the following conditions.
First of all, it has to be a friend of \progclass{boost}\progname{::}\progclass{serialization}\progname{::}\progclass{access}, i.e.\ its class description includes the following line.
\begin{lstlisting}
class my_class
{
    ...
    friend class boost::serialization::access;
    ...
};
\end{lstlisting}
There are two ways of proceeding from here.
Most commonly, the process of saving and loading is the same.
In this situation one defines the template member function
\begin{lstlisting}
template < class Archive >
void serialize( Archive &ar, const unsigned int version );
\end{lstlisting}
whereas the template parameter \progclass{Archive} is filled in by {\bf boost} (and you do not need to know what it is).
Now saving and loading is achieved via the binary operator \progname{\&} applied to \progname{ar} and every member variable we want to save and load.
In the example below, we tell the serialization function, to save and load two out of three member variables.
\begin{lstlisting}
class my_class
{
    ...
    int keep_this;
    int this_is_also_important;
    int useless;
    
    friend class boost::serialization::access;
    template < class Archive >
    void serialize( Archive &ar, const unsigned int )
    {
        ar & keep_this;
        ar & this_is_also_important;
    }
};
\end{lstlisting}

There are rare situations, in which a saving mechanism differs from its loading counter part.
Then one has to implement the function
\begin{lstlisting}
template< class Archive >
void save( Archive & ar, const unsigned int version ) const;
\end{lstlisting}
and its counter part
\begin{lstlisting}
template< class Archive >
void load( Archive & ar, const unsigned int version );
\end{lstlisting}
and one has to tell {\bf boost} to use these two functions by calling the macro
\begin{lstlisting}
BOOST_SERIALIZATION_SPLIT_MEMBER()
\end{lstlisting}
afterwards. A simple example could look like this.
\begin{lstlisting}
class my_class
{
    ...
    int keep_this;
    int this_is_also_important;
    int useless;
    
    friend class boost::serialization::access;
    template< class Archive >
    void save( Archive & ar, const unsigned int ) const
    {
        ar & keep_this & this_is_also_important;
    }
    template< class Archive >
    void load( Archive & ar, const unsigned int )
    {
        ar & keep_this & this_is_also_important;
        useless=0;
    }
    BOOST_SERIALIZATION_SPLIT_MEMBER()
};
\end{lstlisting}
For more information one should consider the online manual of \cite{boost}.
