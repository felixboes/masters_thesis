\section{Organization of our Thesis}

Let us sketch the organization of the content of this thesis. The {\bf first chapter} is this introduction.

The {\bf second chapter} provides a detailed description of our models.
Section \ref{cellular_models:introduction} serves as an overview of our approach.
The details are carried out in the subsequent Sections \ref{cellular_models:from_moduli_spaces_to_parallel_slit_domains}-\ref{cellular_models:ehrenfried}.
Having this done, one has all ingredients to make sense of the following diagram.
\[
    \resizebox{\linewidth}{!}{
        \begin{tikzcd}[row sep=7.5ex, scale=1.3, ampersand replacement=\&]
            \&\&\&\&\&H_\ast(\mathfrak M) \\
            B\Gamma \arrow[<-, shorten <=1ex, shorten >=1ex]{rr}{\simeq} \arrow[dashed, out=25, in=180, shorten <=1ex, shorten >=1ex]{rrrrru}[description]{H_\ast} \&\&
            \mathfrak M \arrow[<-, shorten <=1ex, shorten >=1ex]{rr}[swap]{\text{affine bundle}} \arrow[dashed, shorten <=1ex, shorten >=1ex]{rrru}[description]{H_\ast} \&\&
            \mathfrak H \arrow[shorten <=1ex, shorten >=1ex]{rr}{\cong}[swap]{\text{Hilbert uniformization}} \arrow[dashed, shorten <=1ex, shorten >=1ex]{ru}[description]{H_\ast} \&\&
            \stackbin[\mathfrak R]{\mathfrak P}{\scriptscriptstyle{resp.}} \arrow[<-, shorten <=1ex, shorten >=1ex]{rr}[swap]{\text{Poincaré duality}} \arrow[dashed, shorten <=1ex, shorten >=1ex]{lu}[description]{H_\ast} \&\&
            \stackbin[(R,R')]{(P,P')}{\scriptscriptstyle{resp.}} \arrow[<-, shorten <=1ex, shorten >=1ex]{rr}{\simeq}[swap]{\text{quasi-isomorphic}} \arrow[dashed, shorten <=1ex, shorten >=1ex]{lllu}[description]{H^{\ast-\ldots}} \&\&
            \E \arrow[dashed, out=155, in=0, shorten <=1ex, shorten >=1ex]{lllllu}[description]{H^{\ast-\ldots}}
        \end{tikzcd}
    }
\]
The rightmost model is the Ehrenfried complex.
It is a finite chain complex and its differentials admits an explicit description.
The homology of the moduli spaces is computed via its dual.
In Section \ref{cellular_models:dual_ehrenfried}, we provide an explicit formula for its coboundary operator.
Hereby, we begin with an explanation of our geometric intuition in order to make the upcoming definitions, statements and proofs straightforward.

The {\bf third chapter} is a brief introduction to the cluster spectral sequence.
We introduce the cluster filtration of the bicomplex and the Ehrenfried complex and show that the associated spectral sequences collapse at the second page.

The {\bf fourth chapter} covers various homology operations.
Sections \ref{homology_operations:parallel_patching_slit_pics}-\ref{radial_composition} describe operations defined either for $\Par$ or $\mathfrak{Rad}$.
Operations and formulas relating $\Par$ and $\mathfrak{Rad}$ are discussed in Section \ref{homology_operations:comparision_of_par_and_rad}.
In Section \ref{homology_operations:parallel_patching_slit_pics}, we review well-known operations on $\Par$ via little cubes operads and propose a generalization in Section \ref{homology_operations:glueing_construction}.
Operations on $\Par$ which are induced by bundle maps are discussed in Section \ref{homology_operations:operations_par_via_bundles}.
In Section \ref{homology_operations:alpha}, we present the operations $\alpha$.
The radial multiplication is treated in Section \ref{homology_operations:radial_multiplication} and
the composition of two radial slit domains is reviewed in Section \ref{radial_composition}.
The rotation of radial slit domains in introduced in Section \ref{homology_operations:rotation}.

The {\bf fifth chapter} is a brief analysis of the computational complexity of homology calculations via the Ehrenfried complex.
We compute the number of its cells and discuss nearby algorithms used to derive homological data.

The {\bf sixth chapter} provides the documentation of our software project and lists all cluster spectral sequences we computed.

The {\bf Appendix} reviews possibly unkown notation.