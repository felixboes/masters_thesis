\subsection{The Type HomologyT}
\label{program:libhomology:HomologyT}
In order to derive the homology of a chain complex, we compute all kernels and images of the differentials, given by transposed transformation matrices.
In our situation, we start with a \progclass{ChainComplex} (see Subsection \ref{program:libhomology:ChainComplex}) that is essentially a finite series of matrices of the type \progclass{MatrixT} (see Subsection \ref{program:libhomology:MatrixT}).
The function object \progclass{DiagonalizerT} (see Subsection \ref{program:libhomology:DiagonalizerT}) applies row and column operations until both kernel and image can be read off.
This data should then be communicated to \progclass{HomologyT}.

\subsubsection{Essential Members}
The type \progclass{HomologyT} requires the following members.
You have to provide the types \progclass{HomologyT}\progname{::}\progclass{KernT} respectively \progclass{HomologyT}\progname{::}\progclass{TorsT} that store the kernel respectively the image of a differential.
This can be achieved by including the following two lines in the \progkeyword{public} section of the class definition.
\begin{lstlisting}
class HomologyT{
public:
    typedef /* ... */ KernT;
    typedef /* ... */ TorsT;
};
\end{lstlisting}
You have to define the following two constructors
\begin{lstlisting}
HomologyT ();
HomologyT ( int32_t n, KernT& k, TorsT& t ); // Sets k and t at the spot n.
\end{lstlisting}
and member functions for storing kernels and images.
\begin{lstlisting}
void set_kern ( int32_t , KernT& );
void set_tors ( int32_t , TorsT& );
\end{lstlisting}
Moreover, we want to print the homology to the screen, thus the class definition has to include the following line.
\begin{lstlisting}
friend std::ostream& operator<< ( std::ostream& , const HomologyT& );
\end{lstlisting}

\subsubsection{The Class HomologyDummy}
\label{program:libhomology:HomologyT:HomologyDummy}
As mentioned in Subsection \ref{program:libhomology:DiagonalizerT:DiagonalizerDummy}, there are situations in which one wants to generate a chain complex without computing homological data.
For this purpose, we offer the class
\begin{lstlisting}
class HomologyDummy
\end{lstlisting}
that does absolutely nothing, so you can use it together with \progclass{DiagonalizerDummy} (see Subsection \ref{program:libhomology:DiagonalizerT:DiagonalizerDummy})
as a template parameter for the template class \progclass{ChainComplexT} (see Subsection \ref{program:libhomology:ChainComplex}).

\subsubsection{The Class HomologyField}
Using field coefficients, the homology modules are all vector spaces.
For those who are only interested in the Betti numbers, we offer the class \progclass{HomologyField}.
Here we store only the dimensions of kernel and image.
The class definition is essentially as follows, where the member functions should be self-explaining.
\begin{lstlisting}
class HomologyField{
public:
    typedef int64_t KernT;
    typedef int64_t TorsT;
    
    HomologyField ();
    HomologyField ( int32_t, KernT, TorsT );
    
    void  set_kern   ( int32_t, KernT );
    void  set_tors   ( int32_t, TorsT );
    KernT get_kern   ( int32_t ) const;
    TorsT get_tors   ( int32_t ) const;
    void  erase_kern ( int32_t );
    void  erase_tors ( int32_t );
    friend std::ostream& operator<< ( std::ostream&, const HomologyField& );
private:
    std::map< int32_t, int64_t > kern_rep;
    std::map< int32_t, int64_t > tors_rep;
};
\end{lstlisting}

