\section{Introduction [B,H]}
\label{cellular_models:introduction}
\symbolindex[p]{$\mc P = \{ P_1, \ldots, P_m \}$}{The set of punctures in a Riemann surface $F \in \Modspc$.}{Section \ref{cellular_models:introduction}}
\symbolindex[q]{$\mc Q = (Q_1, \ldots, Q_n)$}{The enumerated points at which the non-vanishing tangent vectors $\mc X$ are attached.}{Section \ref{cellular_models:introduction}}
\symbolindex[x]{$\mc X = (X_1, \ldots, X_n)$}{The enumerated non-vanishing tangent vectors.}{Section \ref{cellular_models:introduction}}
\symbolindex[h]{$\Harmr$}{The bundle of potential functions over $\Modspc$.}{Section \ref{cellular_models:from_moduli_spaces_to_parallel_slit_domains}}
\symbolindex[p]{$\Parr$}{The space of parallel slit domains.}{Sections \ref{cellular_models:introduction} and \ref{cellular_models:parallel}}
\symbolindex[h]{$\mc H$}{The Hilbert uniformization.}{Sections \ref{cellular_models:introduction}, \ref{cellular_models:parallel:bicomplex} and \ref{cellular_models:radial:bisimplicial_complex}}
\symbolindex[p]{$(P,P')$}{The parallel slit complex.}{Sections \ref{cellular_models:introduction} and \ref{cellular_models:parallel}}
\symbolindex[e]{$\E$}{The Ehrenfried complex associated with $(P,P')$ or $(R,R')$}{Sections \ref{cellular_models:introduction} and \ref{cellular_models:ehrenfried}}
\symbolindex[e]{$\E(h,m;r_1, \ldots, r_n)$}{The Ehrenfried complex associated with $(P,P')$ for fixed $g$, $n$, $m$ and $(r_1, \ldots, r_n)$.}{Section \ref{cellular_models:ehrenfried}}
\symbolindex[e]{$\E(h,m)$}{The Ehrenfried complex associated with $(R,R')$ for fixed $g$, $n$ and $m$.}{Section \ref{cellular_models:ehrenfried}}
\symbolindex[r]{$\Re$}{The real part of a complex valued function.}{}
\symbolindex[m]{$\ModspcRad$}{The moduli space of Riemann surfaces with genus $g$, $m$ outgoing boundary curves and $n$ marked incoming boundary curves.}{}
\symbolindex[c]{$\mathcal C^+$}{The set of outgoing boundary curves in a Riemann surface $F \in \ModspcRad$.}{Section \ref{cellular_models:introduction}}
\symbolindex[c]{$ C^+_k$}{A (not distinguishable) outgoing boundary curve in a Riemann surface $F \in \ModspcRad$.}{Section \ref{cellular_models:introduction}}
\symbolindex[c]{$\mathcal C^-$}{The set of incoming boundary curves in a Riemann surface $F \in \ModspcRad$.}{Section \ref{cellular_models:introduction}}
\symbolindex[c]{$ C^-_k$}{The $k\Th$ incoming boundary curve in a Riemann surface $F \in \ModspcRad$.}{Section \ref{cellular_models:introduction}}
\symbolindex[p]{$\mc P = \{ P_1, \ldots, P_n \}$}{The set of marked points on the $n$ incoming boundary curves in a Riemann surface $F \in \ModspcRad$.}{Section \ref{cellular_models:introduction}}
\symbolindex[g]{$\Gamma^\bullet_g(m, n)$}{The mapping class group with respect to $\ModspcRad$.}{}
\symbolindex[h]{$\HarmRad$}{The bundle of potential functions over $\ModspcRad$.}{Section \ref{cellular:radial_bundle}}
\symbolindex[r]{$\Rad$}{The space of radial slit domains.}{Sections \ref{cellular_models:introduction} and \ref{cellular_models:radial}}
\symbolindex[r]{$(R,R')$}{The radial slit complex.}{Sections \ref{cellular_models:introduction} and \ref{cellular_models:radial}}

{\bf [B]} As mentioned above, we are interested in two families of moduli spaces.
The first family consits of the moduli spaces $\Modspc$ parametrizing
Riemann surfaces of genus $g \ge 0$ with $n \ge 1$ boundary curves and $m \ge 0$ permutable punctures.
The second one is composed of moduli spaces $\ModspcRad$ parametrizing Riemann surfaces of genus $g \ge 0$ with $m \ge 1$ permutable outgoing boundary curves and
$n \ge 1$ incoming boundary curves, where on each incoming curve a point is marked.
Let us review Bödigheimer's models for these moduli spaces.
All details as well as valuable pictures are found in \cite{Boedigheimer19901}, \cite{Boedigheimer19902} and \cite{Boedigheimer2006}.

In order to obtain a good semi-multisimplicial model $\Parr$ for $\Modspc$, we replace every boundary curve by a point with a non-zero tangent vector attached.
Thus $\Modspc$ is the moduli space of conformal equivalence classes $[F, \mc P, \mc Q, \mc X]$, where
$F$ is a Riemann surface of genus $g$ with a set of punctures $\mc P = \{ P_1, \ldots, P_m \}$ and with (enumerated) points $\mc Q = ( Q_1, \ldots, Q_n )$ at which (non-vanishing) tangent vectors $\mc X = ( X_1, \ldots, X_n )$ are attached.
The moduli space $\Modspc$ is the quotient of the contractible Teichmüller space by the action of the corresponding mapping class group $\Gamma_{g,n}^m$.
Under the assumption $n \ge 1$, the action of $\Gamma_{g,n}^m$ is free so that $\Modspc$ is a manifold of dimension $6g - 6 + 2m + 4n$.
In this section, we do not elaborate on the advantages of the above mentioned model $\Parr$,
but state that it is primarily used to translate the Dyer-Lashof operations defined on configuration spaces into those defined on the moduli spaces,
see Chapter \ref{homology_operations}.
In what follows, we fix a moduli space $\Modspc$ and an ordered partition $(r_1, \ldots, r_n)$ of $r = r_1 + \ldots + r_n$ with $r_j \ge 1$ for every $j$.

In Section \ref{cellular_models:from_moduli_spaces_to_parallel_slit_domains}, we introduce the bundle $\Harmr$ over $\Modspc$.
The fibre over a point $[F, \mc P, \mc Q, \mc X]$ consists of certain meromorphic 1-forms and a fixed number of integration constants.
It is an open half-space of a real affine space.
Thus the bundle map is a homotopy equivalence. Note that for each partition $(r_1, \ldots, r_n)$ we have such a bundle with fibre dimension depending on $(r_1, \ldots, r_n)$.
The fibre over $[F, \mc P, \mc Q, \mc X]$ parametrizes all harmonic functions $u$ of predescribed behavior, namely with poles of order $r_j$ at
the points $Q_j$ in the direction of $X_j$ and logarithmic sinks at the punctures $P_i$.
Such functions are called potential functions.

The critical gradient flow lines of a given potential function $u$ (along decreasing values of $u$) that leave the critical points of $u$
(to either run into another critical point, into a sink $P_j$ or into a pole $Q_j$) define the critical graph $\mc K_0$ on $F$.
Dissecting $F$ along its critical graph $\mc K_0$, we obtain $r$ open and contractible sub-surfaces $F_1, \ldots, F_r$ which we call basins.
On each basin $F_j$, the harmonic function $u$ is the real part of a holomorphic function $w_j$, unique up to an integration constant.

Each function $w_j$ maps the basin $F_j$ injectively onto an open domain in $\mathbb C$.
Its image is obviously the entire plane $\mathbb C = \mathbb C_j$ with finitely many slits removed,
each slit running from some point horizontally to the left all the way to infinity.
They are commonly called parallel slit domains.
The slits arise from the (piecewise) cuts along the critical flow lines.
Observe that right and left banks of such a flow line are now (piecewise) left respectively right banks of possibly different slits on possibly differrent planes.
Encoding this information allows us to re-glue the surface $F$. 
We define the space $\Parr$ of these parallel slit domains as a subspace of a finite, semi-multisimplicial space $P$,
namely as the complement of the (geometric realization of) the subcomplex of degenerate surfaces $P'$.
The reason for $P$ beeing not multisimplicial is a direct consequence of the fact that the occuring symmetric groups define a semisimplicial set $\Symgrp^\Delta$
together with pseudo degeneracy maps satisfying all but one simplicial identity (compare Appendix \ref{notation:semi_simpl_sym_grp}).

\index{Hilbert uniformization}
Summing up, we have a homeomorphism
\[
    \mc H \colon \Harmr \to \Parr \,,
\]
called Hilbert uniformization.
The pair $(P,P')$ is a relative manifold $\Parr = P - P'$, orientable for $m < 2$ and of dimension $3h$ with $h = 2g+m+n+r-2$.
In particular, the homology of $\Parr$ is given by Poincaré duality
\[
    H_\ast( \Modspc; \mathbb Z ) = H_\ast( \Parr; \mathbb Z ) \cong H^{3h-\ast}( P, P'; \mathcal O)
\]
with $\mathcal O$ the orientation coefficients.
The bicomplex of $(P,P')$ is called parallel slit complex and admits a purely combinatorial description treated in Section \ref{cellular_models:parallel}. 

{\bf [H]} Turning to the second family of moduli spaces $\ModspcRad$,
we consider marked conformal equivalence classes of Riemann surfaces with genus $g \ge 0$ and two kinds of boundary cuves.
There are $m \geq 1$ permutable outgoing boundary curves $C^+_1, \dotsc, C^+_m$ 
and $n \geq 1$ incoming boundary curves $C^-_1, \dotsc, C^-_n$, each of which has a marked point $P_i \in C^-_i$.
Thus, a point in the moduli space $\ModspcRad$ is represented by the data $[F, \mathcal C^+, \mathcal C^-, \mathcal P]$,
where $\mathcal C^+ = C^+_1 \cup \dotsc \cup C^+_m$ respectively $\mathcal C^- = C^-_1 \cup \dotsc \cup C^-_n$ denotes the entire outer respectively inner boundary,
while $\mathcal P = (P_1, \dotsc, P_n)$ is the (enumerated) set of marked points on the $n$ inner boundary curves.
Thereby, a conformal homeomorphism $h \colon F \to F'$ between such surfaces is called marked if each $P_i$ is mapped to $P'_i$,
and the dot in the definition of $\ModspcRad$ refers to this marking.

Analogously to the previous statements,
the moduli space $\ModspcRad$ is a quotient of the contractible Teichmüller space 
under the action of the corresponding mapping class group $\Gamma^\bullet_g(m, n)$.
Excluding the case $g = 0$, $m = n = 1$, the action of $\Gamma^\bullet_g(m, n)$ is free.
Consequently, $\ModspcRad$ is a manifold of dimension $\dim(\ModspcRad) = 6g-6+3m+4n$
and a classifying space $B \Gamma^\bullet_g(m, n)$ for the mapping class group $\Gamma^\bullet_g(m, n)$.

We proceed the same way as in the parallel case: 
In Section \ref{cellular:radial_bundle}, we describe a homotopy equivalent bundle $\HarmRad$ over $\ModspcRad$.
In Section $\ref{cellular_models:radial}$, we introduce the space $\Rad$ of radial slit configurations homeomorphic to $\HarmRad$,
which allows us to actually perform homology calculations.

The bundle $\HarmRad$ is constructed similarly as above:
Over a point $[F, \mathcal C^+, \mathcal C^-, \mathcal P] \in \ModspcRad$, an element of the fibre consists of the data $[F, \mathcal C^+, \mathcal C^-, \mathcal P, w]$ 
with $w = (u, v_1, \dotsc, v_n)$.
By this, $u \colon F \to \overline{\R}$ is a certain harmonic potential function,
and $v_k$ are certain locally defined harmonic conjugates of $u$ on the components, called basins as well, of the complement of the unstable critical gradient flow.
The functions $u$ and $v_k$ are used to map the basins $F_1, \dotsc, F_n$ of $F$ into $n$ annuli $\A_1, \dotsc, \A_n \subset \C$ with outer radius equals $1$.
This process can be reversed.
Since the critical flow lines define segments on the annuli that can be glued together again, the surface $F$ can be re-built from the image of $F$ on the annuli.
The points in the space $\Rad$ of radial slit domains consist of these annuli together with the information neccessary to encode the glueing.
We define $\Rad$ in a simplicial way
such that $\Rad = R - R'$ is the geometric realization of the complement of a subcomplex $R'$ of the semi-multisimplicial complex $R$.

The pair $(R, R')$ is a relative manifold of dimension $3h + n$ with $h = 2g - 2 + m + n$, which is orientable for $m < 2$.
Altogether, we can express the homology of $\ModspcRad$ via Poincaré duality by
\[
    H_\ast( \ModspcRad; \mathbb Z ) = H_\ast( \Rad; \mathbb Z ) \cong H^{3h+n-\ast}( R, R'; \mathcal O)\,,
\]
where $\mathcal O$ denotes the orientation coefficients.
In both Sections \ref{cellular_models:parallel} and \ref{cellular_models:radial}, we give a description of the cells of the bicomplexes $P$ and $R$,
followed by the explanation of the vertical and horizontal faces, yielding the boundary operators.
In Section \ref{cellular_models:orientation}, we define the orientation system $\mathcal O$ for the bicomplexes $P$ and $R$.

{\bf [B]} We are left to determine the (co)homology of the finite bicomplex $(P,P')$ respectively $(R,R')$.
Both complexes have a close connection to the study of the homology of normed groups and in particular the symmetric groups.
To see this, let $G$ be a group with a norm $N$.
The bar complex $B_\bullet(G)$ can be filtered by extending the norm for an element $\inhomq[g]$ by
$N(g_q | \ldots | g_1) = N(g_q) + \ldots + N(g_1)$.
In \cite{Visy201011}, the spectral sequence $\mc N[G]$ associated with this norm fltration on $B_\bullet(G)$ is studied for a family of groups called factorable (c.f.\ Appendix \ref{notation:factorable_groups}).
These are normed groups with a certain normal form for its elements in a given set of generators whose word length norm is $N$.
The symmetric groups with all transpositions as generators are examples of factorable groups.
Other examples are more general Coxeter groups.
The main result in \cite{Visy201011} states that for a factorable group, the first page of the above spectral sequence is concentrated in a single diagonal.
Thus the homology of $G$ is the homology of this diagonal.
It turns out that the $p\Th$ column of our bicomplex $(P,P')$ respectively $(R,R')$ is a direct summand of the $h\Th$ column of $\mc N^0[\Symgrp_p]$,
where $q$ is the homological degree of both sides.
It follows that the columns of our bicomplex have their homology concentrated in the degree $q=h$,
so the first page of the spectral sequence of the double complex is concentrated in a single row which is by definition the Ehrenfried complex $\E$ associated with $(P,P')$ respectively $(R,R')$.

In Section \ref{cellular_models:ehrenfried}, we adapt the known methods to show that $\E$ is a quasi-isomorphic direct summand of $(P,P')^\ast$ respectively $(R,R')^\ast$,
which was already formulated and proven for the parallel case with $n=1$ and $r=1$ the trivial partition.

In Section \ref{cellular_models:dual_ehrenfried}, we give an explicit description of the dual Ehrenfried complex:
We elaborate our geometric insights on the behavior of the horizontal coface operator, in order to introduce the notion of $i\Th$ coboundary traces.
We then define a canonical bijection
between the set of $i\Th$ coboundary traces and the set of $i\Th$ cofaces.
The coface of a top dimensional cell $\Sigma$ corresponding to a given coboundary trace $a$ will be called $a.\Sigma$.
In Proposition \ref{cellular_models:dual_ehrenfried:cob_are_equal_via_sigma_j_i}, we provide the formula
\[
    \del_\E^\ast(\Sigma) = \sum_{i=1}^{p-1} (-1)^i \sum_{a \in T_i(\Sigma)} \kappa^\ast(a.\Sigma)\,,
\]
which enables us to define the homology operations described in \cite{Boedigheimer19902} and \cite{Boedigheimer201314} via the dual Ehrenfried complex in Chapter \ref{homology_operations}.
Another benefit of this definition is the classification of the cells of the Ehrenfried complex (c.f.\ Proposition \ref{cellular_models:dual_ehrenfried:every_cell_is_an_expansion}).
Roughly speaking, there are a few distinguished cells of low degree which we call thin.
An arbitrary cell is uniquely obtained from such a thin cell by glueing in a certain number of stripes.
The geometric meaning and the precise statements are presented in Subsection \ref{cellular_models:dual_ehrenfried:classification_of_the_cells}.

