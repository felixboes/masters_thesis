\section{Rotation of Radial Slit Pictures [H]}
\label{homology_operations:rotation}

In this section, we want to construct an operation on the space $\Rad$ of radial slit domains that looks like a rotation of the annuli on which radial slit pictures reside.

Consider the bundle $\HarmRad \cong \Rad$ first.
Recall that its elements are tuples $[F, \mathcal C^+, \mathcal C^-, \mathcal P, w]$.
For an explanation of the notation and more details about the following explanation, see Section \ref{cellular:radial_bundle}.
Here, we only mention that $\mathcal P = (P_1, \dotsc, P_n)$ is the set of marked points on the enumerated incoming boundary curves $C^-_1, \dotsc, C^-_n$ of the surface $F$, 
On each incoming boundary curve $C^-_k$, we can let the sphere act by rotating the marked point $P_k$ by an angle $\theta_k$.
We obtain

\begin{defi}
   There is a group operation
  \[
    \psi \colon (\mathbb S^1)^n \times \HarmRad \to \HarmRad
  \]
  given by
  \[
    (\theta_1, \dotsc, \theta_n).[F, \mathcal C^+, \mathcal C^-, (P_1, \dotsc, P_n), w] = [F, \mathcal C^+, \mathcal C^-, (\theta_1 P_1, \dotsc, \theta_n P_n), w]\,.
  \]
  Here, $\theta_k P_k$ is a shorthand for 
  \[
  \theta_k P_k = \varphi_k(\theta_k\varphi_k^{-1}(P_k)) \in C^-_k\,,
  \]
  where $\varphi_k \colon \mathbb S^1 \to C^-_k$ parametrizes the $k\Th$ incoming boundary curve.
\end{defi}

Recall that we obtain a radial slit picture from $[F, \mathcal C^+, \mathcal C^-, \mathcal P, w]$ by mapping it into $n$ annuli $\A_1, \dotsc, \A_n$,
whereby the marked point $P_k$ is mapped to the real point of the inner boundary of the annulus $\A_k$.
Hence, a rotation of $P_k$ by an angle $\theta_k$ implies a rotation of the marked point on $\A_k$ by $\theta_k$.
Equivalently, we can leave the marked point on the annulus in the same place as before and rotate all slits on the annulus by an angle $-\theta_k$ instead.
We chose the latter variant in order to avoid confusions about the location of the marked point.
\begin{figure}[ht]
  \centering
  \incgfx{pictures/rotation}
  \caption{\label{rotation} Rotating a radial slit picture by an angle of $50$ degree.}
  \end{figure}
By composing the induced map of this group operation in homology with the homology cross product, we obtain

\begin{defi}
   There is a map 
   \[
       H_i((\mathbb S^1)^n) \otimes H_j(\HarmRad) \to H_{i+j}(\HarmRad)\,, \gamma \otimes x \mapsto \psi_*(\gamma \times x)\,.
   \]
   The restriction 
   \[
       \rot_\gamma \colon H_j(\HarmRad) \to H_{j+1}(\HarmRad)\,, \rot_\gamma = \psi_*(\gamma \times x)
   \]
   of this map is called the \textbf{rotation map} with respect to $\gamma \in H_1((\mathbb S_1)^n)$.
\end{defi}

\begin{prop}
   The rotation map satisfies the following properties:
   \begin{enumerate}
    \item Rotation is associative, i.e. for $\gamma_1, \gamma_2 \in H_1((\mathbb S_1)^n)$, we have
    \[
       \rot_{\gamma_1} \rot_{\gamma_2} = \rot_{\gamma_1 \cdot \gamma_2}\,.
    \]
    Here, $\_\cdot\_$ denotes the Pontryagin product.
    \item Rotation defines a differential of degree $+1$ on $H_*(\HarmRad)$, i.e. $\rot_\gamma^2 = 0$ for $\gamma \in H_1(\mathbb S_1)$.
   \end{enumerate}
    \begin{proof}
       The first statement is obvious since $\psi$ is a group operation and the cross product is an isomorphism in our case.
       The second claim follows from the first one since the ring $H_*((\mathbb S^1)^k)$ is strictly anticommutative, i.e.,
       $\gamma^2 = 0$ if the degree of $\gamma$ is odd, which is the case here.
       Hence,
       \[
           \rot_\gamma^2 = \rot_{\gamma^2} = \rot_0 = 0\,. 
       \]
       
    \end{proof}
\end{prop}

