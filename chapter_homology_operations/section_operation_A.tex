\section{The Operation \texorpdfstring{$\alpha$}{alpha} [H]}
\label{homology_operations:alpha}
During this section, let $n = r = 1 + \dotsb + 1$ the trivial partition.

Consider a surface $F$ with genus $g$, $m$ punctures and $n$ boundary curves, and choose the $k\Th$ of these boundary curves. 
We can then add another puncture to $F$ by glueing a cylinder with a puncture to this boundary curve,
see Figure \ref{map_A_on_surfaces}.

\begin{figure}[ht]
\centering
\def\svgwidth{0.3\columnwidth}
\input{pictures/map_A_on_surfaces.pdf_tex}
\caption{\label{map_A_on_surfaces} Adding a puncture to a surface.}
\end{figure}

In order to realize this map on parallel slit pictures, consider the $k\Th$ level of a parallel slit picture $L$.
Recall that the border of the relevant clipping of $L$ can be identified with the boundary curves and punctures of $L$. 
The $k\Th$ boundary curve needs to be seperated into one new boundary curve and one new puncture.
Therefore, we insert a new pair of slits per level as in Figure \ref{map_A_on_cells}.
The new slits are longer than any other slit of the slit picture, and they are placed below respectively above all other slits.

\begin{figure}[ht]
\centering
\incgfx{pictures/map_A_on_cells}
\caption{\label{map_A_on_cells} Adding a puncture to a parallel slit picture.}
\end{figure}

In \cite[Theorem 1.3]{BoedigheimerTillmann2001}, Bödigheimer and Tillmann prove the following

\begin{thm}
   The map $\alpha_k \colon \Modspc \to \mathfrak{M}^{m+1}_{g,n}$ admits a stable retraction.
   In particular, $\alpha_k$ induces an injective splitting in homology.
\end{thm}

By applying the map $\alpha_k$ successively for all $k = 1, \dotsc, n$, we obtain 

\begin{defprop}
\label{alpha}
\symbolindex[a]{$\alpha$}{The map $\alpha$ adds $n$ new punctures to a parallel cell on $r = n$ levels.}{Definition \ref{alpha}}
The map
\[
   \alpha = \alpha_n \circ \dotsc \circ \alpha_1 \colon \Modspc \to \mathfrak{M}^{m+n}_{g,n}
\]
induces a split injective map
\[
   \alpha_* \colon H_*(\Modspc) \to H_*(\mathfrak{M}^{m+n}_{g,n})
\]
in homology.
\end{defprop}

As promised in Subsection \ref{cellular_radialization}, we can now proof

\begin{cor}\label{cor_rad_injective}
   The radialization map 
   \[
      \radmap \colon \Modspc \to \ModspcRad
   \]
   induces a split injective map on homolgoy.
   \begin{proof}
       Note that 
       \[
          \alpha = \parmap \circ \radmap\,,
       \]
       which can be seen in Figure \ref{rad_injective}.
       \begin{figure}[ht]
	\centering
	\def\svgwidth{\columnwidth}
	\input{pictures/rad_injective.pdf_tex}
	\caption{\label{rad_injective} Applying radialization and then parallelization to a surface with punctures and boundaries.}
	\end{figure}
	The claim follows with Proposition \ref{alpha}.
   \end{proof}
\end{cor}
