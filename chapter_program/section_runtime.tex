\section{Runtime and Memory Improvements [H]}
\label{program:runtime}

Recall that the enormous size of the Ehrenfried complex forces us to take care of the runtime and memory consumption of our computer program (compare Section \ref{complexity:number_of_mono_cells}).
Via some example calculations, we show how we have improved the performance of our computer program step by step.

Consider the moduli spaces $\mathfrak M_{3, 1}^1$ and $\mathfrak M_3(m, n)$. 
The $E^0$ term of the corresponding cluster spectral sequence on the parallel Ehrenfried complex looks as follows:

\begin{center}
  \begin{tabular}{r||r|r|r|r|r|r|r|}
      \cline{2-8}
      \multicolumn{1}{r|}{} & \multicolumn{7}{c|}{$g = 3$, $m = 1$: Parallel $E^0_{p,l}$ for $\mathbb F_2$} \\ \hline
      \tl{\diagbox[height=1.7em, width=3em]{$p$}{$l$}} & 1 & 2 & 3 & 4 & 5 & 6 & 7\\ \hline\hline
      \tl 2   & 1     &       &       &       &       &       & \\ \hline
      \tl 3   & 252   &       &       &       &       &       & \\ \hline
      \tl 4   & 7563  & 18    &       &       &       &       & \\ \hline
      \tl 5   & 81360 & 2010  &       &       &       &       & \\ \hline
      \tl 6   & 424920& 48855 & 195   &       &       &       & \\ \hline
      \tl 7   &1141056& 469938& 13230 &       &       &       & \\ \hline
      \tl 8   &1305876&2069844& 247898& 1540  &       &       & \\ \hline
      \tl 9   &       &3593880&1810368& 70476 &       &       & \\ \hline
      \tl{10} &       &       &4737360& 915390& 8715  &       & \\ \hline
      \tl{11} &       &       &       &3702820& 258720&       & \\ \hline
      \tl{12} &       &       &       &       &1765335& 31878 & \\ \hline
      \tl{13} &       &       &       &       &       & 477906& \\ \hline
      \tl{14} &       &       &       &       &       &       & 56628 \\ \hline
  \end{tabular}
\end{center}

We were not able to calculate the entire homology of $\mathfrak M_{3, 1}^1$ via this spectral sequence since there are a lot of modules with several million of basis elements,
exeeding our possibilities concerning runtime and memory.
But we can use the radial Ehrenfried complex cluster spectral sequence for our calculations instead due to Proposition \ref{cellular_models:comparision_of_the_models:bundles_are_h_equiv}.
This reduces the dimensions of the modules of the $E^0$ page enourmously:

\begin{center}
  \begin{tabular}{r||r|r|r|r|r|r|r|}
      \cline{2-8}
      \multicolumn{1}{r|}{} & \multicolumn{7}{c|}{$g = 3$, $m = 1$: Radial $E^0_{p,l}$ for $\mathbb F_2$} \\ \hline
      \tl{\diagbox[height=1.7em, width=3em]{$p$}{$l$}} & 1 & 2 & 3 & 4 & 5 & 6 & 7 \\ \hline \hline
      \tl 1   & 1     &       &        &       &       &       &      \\  \hline
      \tl 2   & 82    & 1     &        &       &       &       &      \\  \hline
      \tl 3   & 1212  & 91    &        &       &       &       &      \\  \hline
      \tl 4   & 7200  & 1652  & 9      &       &       &       &      \\  \hline
      \tl 5   & 20400 & 12890 & 500    &       &       &       &      \\  \hline
      \tl 6   & 23760 & 49380 & 7706   & 60    &       &       &      \\  \hline
      \tl 7   &       & 77924 & 48104  & 2310  &       &       &      \\  \hline
      \tl 8   &       &       & 111588 & 25676 & 294   &       &      \\  \hline
      \tl 9   &       &       &        & 91384 & 7497  &       &      \\  \hline
      \tl{10} &       &       &        &       & 44850 & 945   &      \\  \hline
      \tl{11} &       &       &        &       &       & 12375 &      \\  \hline
      \tl{12} &       &       &        &       &       &       & 1485 \\  \hline
  \end{tabular}
\end{center}

The biggest module on the first page for the radial case has dimension $111588$, and not dimension $4737360$ as in the parallel case.
Using all our strategies to improve the performance of our program,
we can now determine the homology of $\mathfrak M_{3, 1}^1$ and $\mathfrak M_3(m, n)$ within less than half an hour.
In each row of the following table, we see running time and maximum memory consumption of one run of our computer program with different improvement techniques. 

\begin{center}
  \begin{tabular}{|c|c|c|l||r|r|}
      \cline{1-6}
      \multicolumn{6}{|c|}{Runtime and Memory Results for $g = 3$, $m = 1$} \\ \hline \hline
      Radial & CSS & Bool & \# Threads            & Runtime [h:min:sec] & Max. Memory Used [MB] \\ \hline\hline
      x      & x   & x    & $t_w = 8$, $t_r = 4$  & $0:27:43$           & $7056$  \\ \hline
      x      & x   & x    & $t_w = 11$, $t_r = 1$ & $0:39:03$           & $7071$  \\ \hline
      x      & x   & x    & $t = 1$               & $1:37:49$           & $7031$  \\ \hline
      x      &     & x    & $t_w = 8$, $t_r = 4$  & $1:25:39$           & $10819$ \\ \hline
      x      & x   &      & $t_w = 8$, $t_r = 4$  & $18:36:22$          & $92857$ \\ \hline
             & x   & x    & $t_w = 8$, $t_r = 4$  & /                   & /       \\ \hline
  \end{tabular}
\end{center}

Thereby, a cross marks whether 
\begin{itemize}
\item we use the radial Ehrenfried complex or the parallel one, 
\item we filter this by cluster sizes,
\item we use our special implementation of boolean coefficients (compare Subsubsection \ref{program:libhomology:MatrixT:MatrixField_for_F_2_and_css}),
\end{itemize}
and we specify whether we run the program single threaded ($t = 1$) or parallel, and if parallel, 
how many threads we use as working threads ($t_w$) and as remaining threads ($t_r$).
For an explanation of the meaning of these threads, see Subsubsection \ref{program:libhomology:DiagonalizerT:DiagonalizerField}.

We see that the most significant effect on both runtime and memory consumption arises by implementing boolean coefficients in a clever way
-- the runtime decreases by a factor $40$ and memory by a factor $13$.
Using this implementation of boolean coefficients, another important step to improve performance is to filter the Ehrenfried complex by cluster sizes,
which yields another factor $3$ of runtime and a factor $1.5$ of memory consumption improvement.
Unfortunately, parallelizing the program with $12$ threads does not gain another factor $12$ concerning runtime 
since diagonalizing is not smoothly parallelizable.
Still, another factor $2.5$ of runtime improvement, when only the jobs of the so-called working threads are distributed,
and even a factor of $3.6$, when also the so-called remaining work is distributed, results in valuable reduction of running time.