\subsection{The Coface Operator via Coboundary Traces [B]}
In this subsection, we proceed as described above.
We encode our geometric intuition in the notion of $i\Th$ coboundary trace and show
that the canonical map from the set of $i\Th$ coboundary traces $T_i(\Sigma)$ to the set of $i\Th$ cofaces $\cof_i(\Sigma)$ is bijective.
Hereby, we focus on the parallel Ehrenfried complex as our proofs become more compact and the radial case is treated analogously.

Recall that the horizontal boundary operator is defined as the alternating sum of the faces of $\Delta^{p_1} \times \ldots \times \Delta^{p_r}$.
It is therefore computed ``levelwise'' and it might be helpful to think of parallel slit domains with exactly one level.

\begin{defi}
    \label{cellular_models:dual_ehrenfried:cofaces}
    \index{coboundary!coboundary of a cell}
    \index{coboundary!set of coboundaries}
    \symbolindex[c]{$\cof_i(\Sigma)$}{The set of all $i\Th$ cofaces of a given cell $\Sigma$}{Definition \ref{cellular_models:dual_ehrenfried:cofaces}}
    Let $\Sigma = \inhom$ be a non-degenerate cell in the double complex $P/P'$ of bidegree $(p,h)$.
    The set of its $i\Th$ cofaces is denoted by
    \[
        \cof_i(\Sigma) = \{ \tilde\Sigma \in P_{p+1, h} \mid d_i''(\tilde\Sigma) = \Sigma \} \,.
    \]
\end{defi}

\begin{defi}
    \label{cellular_models:dual_ehrenfried:cob_tr}
    \index{coboundary!coboundary trace}
    \index{coboundary!set of coboundary traces}
    \symbolindex[a]{$\homog[a]$}{A coboundary trace of a cell $\Sigma$.}{Definition \ref{cellular_models:dual_ehrenfried:cob_tr}}
    \symbolindex[t]{$T_i(\Sigma)$}{The set of all $i\Th$ coboundary traces of a given cell $\Sigma$.}{Definition \ref{cellular_models:dual_ehrenfried:cob_tr}}
    Let $\Sigma = \inhom$ be a parallel non-degenerate top dimensional cell with respect to $[p]$ and let $\ul 0_k < i \le \ul p_k$.
    A sequence $\homog[a]$ in $[p]$ is called $i\Th$ {\bf coboundary trace of $\Sigma$} if it satisfies the following conditions:
    \begin{enumerate}
        \item \label{cellular_models:dual_ehrenfried:cob_tr:normalization} $a_0 = i+1$.
        \item \label{cellular_models:dual_ehrenfried:cob_tr:choice} If $a_j \neq a_{j-1}$, then $a_j = (S_i\tau_j)(a_{j-1})$
                    (or equivalently: if $a_j \neq (S_i\tau_j)(a_{j-1})$, then $a_j = a_{j-1}$).
        \item \label{cellular_models:dual_ehrenfried:cob_tr:i} $a_j \neq (S_i\tau_j)(a_{j-1})$ at least once.
        \item \label{cellular_models:dual_ehrenfried:cob_tr:i+1} $a_j \neq a_{j-1}$ at least once.
    \end{enumerate}
    The set of all $i\Th$ coboundary traces of $\Sigma$ is denoted by
    \[
        T_i(\Sigma) = \{ (a_h : \ldots : a_0) \text{ is an $i\Th$ coboundary trace of $\Sigma$} \} \,.
    \]
\end{defi}

Let us elaborate on the above definition.
Condition \ref{cellular_models:dual_ehrenfried:cob_tr:normalization} is the normalization corresponding to $\sigma_0(i) = i+1$.
The second condition encodes jumping slits and glueing in stripes.
In this sense, condition \ref{cellular_models:dual_ehrenfried:cob_tr:i} forbids glueing in a stripe below all stripes of height $i$ and
condition \ref{cellular_models:dual_ehrenfried:cob_tr:i+1} forbids glueing in a stripe above all stripes of height $i$.
Recalling that the $\ul0_k\Th$ slit of a radial cells might be empty the next definition is the obvious analogue to Definition \ref{cellular_models:dual_ehrenfried:cob_tr}.

\begin{defi}
    \label{cellular_models:dual_ehrenfried:cob_tr_radial}
    Let $\Sigma = \inhom$ be a radial non-degenerate top dimensional cell with respect to $[p]$ and let $\ul 0 < i \le \ul{p+1}_k$.
    A sequence $\homog[a]$ in $[p]$ is called $i\Th$ {\bf coboundary trace of $\Sigma$} if it satisfies the following conditions:
    \begin{enumerate}
        \item \label{cellular_models:dual_ehrenfried:cob_tr_radial:normalization} $a_0 = \begin{cases}i+1& i \neq \ul{p+1}_k \\ \ul0_k & i = \ul {p+1}_k\end{cases}$.
        \item \label{cellular_models:dual_ehrenfried:cob_tr_radial:choice} If $a_j \neq a_{j-1}$, then $a_j = (S_i\tau_j)(a_{j-1})$
                    (or equivalently: if $a_j \neq (S_i\tau_j)(a_{j-1})$, then $a_j = a_{j-1}$).
        \item \label{cellular_models:dual_ehrenfried:cob_tr_radial:i} If $i \neq \ul0_k$ we have $a_j \neq (S_i\tau_j)(a_{j-1})$ at least once.
        \item \label{cellular_models:dual_ehrenfried:cob_tr_radial:i+1} If $i \neq \ul{p+1}_k$ we have $a_j \neq a_{j-1}$ at least once.
    \end{enumerate}
    The set of all $i\Th$ coboundary traces of $\Sigma$ is denoted by
    \[
        T_i(\Sigma) = \{ (a_h : \ldots : a_0) \text{ is an $i\Th$ coboundary trace of $\Sigma$} \} \,.
    \]    
\end{defi}


\begin{rem}
    \label{cellular_models:dual_ehrenfried:cob_tr_do_not_contain_i}
    Observe that the symbol $i$ does not occur in an $i\Th$ coboundary trace.
\end{rem}

\begin{defi}
    \label{cellular_models:dual_ehrenfried:a_dot_sigma}
    \symbolindex[a]{$a.\Sigma$}{The coboundary of $\Sigma$ corresponding to $a$.}{Definition \ref{cellular_models:dual_ehrenfried:a_dot_sigma}}
    Let $\Sigma = \inhom$ be a non-degenerate top dimensional cell with respect to $[p]$ and let $i = \ul i_k \in [p]$ and $a \in T_i(\Sigma)$.
    Then we define
    \[
        a.\Sigma = \inhom[\tilde\tau]
    \]
    with
    \[
        \tilde\tau_j =
            \begin{cases}
                S_i\tau_j                               & a_j = (S_i\tau_j)(a_{j-1}) \\
                (i\ a_{j-1}) S_i\tau_j (i\ a_{j-1})     & a_j \neq (S_i\tau_j)(a_{j-1})
            \end{cases}
    \]
    and with respect to the partition of $p+1 = p_1 + \ldots + p_{k-1} + (p_k + 1 ) + p_{k+1} + \ldots + p_r$.
\end{defi}

\begin{rem}
    \label{cellular_models:dual_ehrenfried:tilde_tau_i_iS_i}
    For $a \in T_i(\Sigma)$ and $a.\Sigma = \inhom[\tilde\tau]$, we have
    \[
        a_j \neq (S_i\tau_j)(a_{j-1}) \mspc{iff}{10} \tilde\tau_j(i) \neq i \mspc{iff}{10} \tilde\tau_j = (i\ (S_i\tau_j)(a_{j-1})) \,.
    \]
\end{rem}

\begin{prop}
    \label{cellular_models:dual_ehrenfried:cob_tr_equals_cob}
    Consider $P=P(h,m;r_1, \ldots, r_n)$, the bisimplicial complex associated with $\Modspc$.
    For every non-degenerate cell $\Sigma \in P_{p,h}$ and $\ul1_k \le i \le \ul p_k$, the map
    \[
        \Phi \colon T_i(\Sigma) \to \cof_i(\Sigma) \mspc{with}{20} a \mapsto a.\Sigma
    \]
    is bijective.
    The $p\Th$ coboundary operator of the associated Ehrenfried complex is therefore
    \[
        \del_\E^\ast( \Sigma ) = \sum_{i=1}^p (-1)^i \sum_{a \in T_i(\Sigma)} \kappa^\ast(a.\Sigma) \,.
    \]
\end{prop}

We prove this proposition using the following basic properties.

\begin{lem}
    \label{cellular_models:dual_ehrenfried:cob_are_equal_via_sigma_j_i}
    Consider $\Sigma = (\tau_h \mid \ldots \mid \tau_1)$ and $\Sigma' = (\tau_h' \mid \ldots \mid \tau_1')$ of bidegree $(p,h)$
    that have their $i\Th$ horizontal face in common.
    Assume that $\sigma_j(i) = \sigma'_j(i)$ for all $j$.
    Then already $\Sigma = \Sigma'$.
\end{lem}

\begin{proof}
    Using the definition of the horizontal differential (see \ref{cellular_models:parallel:faces}), we have
    \[
        (i\ \sigma_j(i)) \sigma_j = D_i(\sigma_j) = D_i(\sigma'_j) = (i\ \sigma'_j(i)) \sigma'_j = (i\ \sigma_j(i)) \sigma'_j
    \]
    for arbitrary $j$, up to renormalization.
    Hence $\sigma_j = \sigma'_j$ for all $j$ and the claim follows.
\end{proof}

\begin{lem}
    \label{cellular_models:dual_ehrenfried:cob_tr_def_sigma_j_i}
    Let $a = \homog[a]$ be an $i\Th$ coboundary trace of $\Sigma$ and denote $a.\Sigma = \homog[\tilde\sigma]$.
    Then we have
    \[
        a_j = \tilde\sigma_j(i) \,.
    \]
\end{lem}

\begin{proof}
    By construction $a_0 = i+1 = \tilde\sigma_0(i)$.
    We assume there is a minimal index $j$ with $a_j \neq \tilde\sigma_j(i)$.
    Hence
    \[
        a_j \neq \tilde\sigma_j(i) =
            \tilde\tau_j(\tilde\sigma_{j-1}(i)) =
            \tilde\tau_j(a_{j-1}) =
            \begin{cases}
                (S_i\tau_j)(a_{j-1})                                    & a_j = (S_i\tau_j)(a_{j-1}) \\
                \big( (i\ a_{j-1}) S_i\tau_j (i\ a_{j-1}) \big)(a_{j-1}) = a_{j-1}  & a_j \neq (S_i\tau_j)(a_{j-1})
            \end{cases}
    \]
    by definition of $\tilde\tau_j$.
    The first case is clearly impossible, and the second case implies $\sigma_j(i) = a_{j-1} = a_j$ by \ref{cellular_models:dual_ehrenfried:cob_tr:choice} in Definition \ref{cellular_models:dual_ehrenfried:cob_tr}.
\end{proof}

\begin{lem}
    \label{cellular_models:dual_ehrenfried:a_Sigma_is_cob}
    Let $a \in T_i(\Sigma)$.
    Then
    \[
        d_i''(a.\Sigma) = \Sigma \,.
    \]
\end{lem}

\begin{proof}
    Denote $a.\Sigma = \inhom[\tilde\tau] = \homog[\tilde\sigma]$ and $d_i''(a.\Sigma) = \inhom[\tilde\tau'']$.
    For $q \ge j \ge 1$, by Proposition \ref{cellular_models:parallel:prop_dh} and Lemma \ref{cellular_models:dual_ehrenfried:cob_tr_def_sigma_j_i},
    we have
    \[
        \tilde\tau_j'' = D_i( \tilde\tau_j \cdot (i\ a_{j-1})) = 
            \begin{cases}
                D_i( S_i\tau_j \cdot (i\ a_{j-1})) & a_j = (S_i\tau_j)(a_{j-1}) \\
                D_i( (i\ a_{j-1}) \cdot S_i\tau_j) & a_j \neq (S_i\tau_j)(a_{j-1}) \\
            \end{cases} \,.
    \]
    Now $(i\ a_{j-1})$ is disregarded in both cases by Proposition \ref{cellular_models:parallel:prop_dh} and we are done as $D_iS_i\tau_j = \tau_j$.
\end{proof}

\begin{lem}
    \label{cellular_models:dual_ehrenfried:a_Sigma_is_non_deg}
    Consider a non-degenerate cell $\Sigma \in P_{p,h}$ and let $a \in T_i(\Sigma)$.
    Then the cell $a.\Sigma \in P_{p+1,h}$ is also non-degenerate.
\end{lem}

\begin{proof}
    Denote
    \[
        \Sigma = \inhom = \homog \mspc{and}{20} a.\Sigma = \tilde\Sigma = \inhom[\tilde\tau] = \homog[\tilde\sigma] \,.
    \]
    We show that $\tilde\Sigma$ is a connected inner cell with the correct number of punctures and boundaries.
    
    Clearly, $\tilde\tau_j \neq 1$ for all $q \ge j \ge 1$.
    Recall that the $\tilde\tau_q, \ldots, \tilde\tau_1$ have a fixed point $k$ in common if and only if $\tilde\sigma_j(k-1) = k$ for all $j$.
    By assumtion, $D_i(\tilde\sigma_j) = \sigma_j$ and $\Sigma$ is an inner cell.
    Hence, it suffices to check $k=i,i+1$.
    By condition \ref{cellular_models:dual_ehrenfried:cob_tr:i} in Definition \ref{cellular_models:dual_ehrenfried:cob_tr},
    there is at least one $a_j \neq (S_i\tau_j)(a_{j-1})$.
    This implies $\tilde\tau_j(i) \neq i$ (see Remark \ref{cellular_models:dual_ehrenfried:tilde_tau_i_iS_i}).
    By condition \ref{cellular_models:dual_ehrenfried:cob_tr:i+1} in Definition \ref{cellular_models:dual_ehrenfried:cob_tr},
    there is at least one $j$ with $a_j \neq a_{j-1}$.
    This implies $\tilde\sigma_j(i) \neq i+1$ for at least one $j$.
    Thus $\tilde\Sigma$ is an inner cell.
    
    By Lemma \ref{cellular_models:dual_ehrenfried:a_Sigma_is_cob}, $\tilde\Sigma$ is an $i\Th$ coboundary of $\Sigma$.
    In particular, it is connected.
    Moreover, $\tilde\sigma_j(i) = a_j \neq i$ (by Lemma \ref{cellular_models:dual_ehrenfried:cob_tr_def_sigma_j_i} and Remark \ref{cellular_models:dual_ehrenfried:cob_tr_do_not_contain_i}) 
    and $D_i(\tilde\sigma_j) = \sigma_j$.
    Therefore
    \[
        \ncyc(\tilde\Sigma) = \ncyc(\Sigma)
    \]
    and, by construction, $N(\tilde\Sigma) = h$.
    The levels of $\tilde\Sigma$ are ordered ascendingly as this is true for $\Sigma$.
\end{proof}

\begin{proof}[Proof of Proposition \ref{cellular_models:dual_ehrenfried:cob_tr_equals_cob}]
    The map $\Phi$ is well defined by Lemma \ref{cellular_models:dual_ehrenfried:a_Sigma_is_non_deg} and Lemma \ref{cellular_models:dual_ehrenfried:a_Sigma_is_cob}.
    
    By Lemma \ref{cellular_models:dual_ehrenfried:cob_tr_def_sigma_j_i},
    every $i\Th$ coboundary trace $a\in T_i(\Sigma)$ defines a coboundary $a.\Sigma = \homog[\tilde\sigma]$ with $a_j = \tilde\sigma_j(i)$.
    We conclude that $\Phi$ is injective (as a consequence of Lemma \ref{cellular_models:dual_ehrenfried:cob_are_equal_via_sigma_j_i}).
    
    Using Lemma \ref{cellular_models:dual_ehrenfried:cob_tr_def_sigma_j_i} and Lemma \ref{cellular_models:dual_ehrenfried:cob_are_equal_via_sigma_j_i},
    it remains to show that every $i\Th$ coboundary $\tilde\Sigma = \homog[\tilde\sigma]$ defines an $i\Th$ coboundary trace $a$ of $\Sigma$ with $a_j = \tilde\sigma_j(i)$.
    Both the conditions
    \[
        \tilde\sigma_0(i) = i+1 \mspc{and}{20} \tilde\sigma_j(i) \neq \tilde\sigma_{j-1}(i) \text{ at least once}
    \]
    are clearly satisfied; it remains to prove \ref{cellular_models:dual_ehrenfried:cob_tr:choice} and \ref{cellular_models:dual_ehrenfried:cob_tr:i}.
    
    In order to show condition \ref{cellular_models:dual_ehrenfried:cob_tr:choice} of Definition \ref{cellular_models:dual_ehrenfried:cob_tr},
    let $\tilde\sigma_{j-1}(i) \neq \tilde\sigma_j(i) = \tilde\tau_j(\tilde\sigma_{j-1}(i))$.
    But $\tilde\sigma_{j-1}(i) \neq i \neq \tilde\sigma_j$ by Proposition \ref{cellular_models:ehrenfried:cor_d_hor_deg}, hence $\tau_j(i) = i$.
    Now, by Proposition \ref{cellular_models:parallel:prop_dh},
    \[
        \tau_j = D_i(\tilde\tau_j (i\ \tilde\sigma_{j-1}(i))) = D_i( \tilde\tau_j)\,, \mspc{hence}{20} \tilde\tau_j = S_i\tau_j \,.
    \]
    We have shown that
    \[
        \tilde\sigma_j(i) \neq \tilde\sigma_{j-1}(i) \mspc{implies}{20} \tilde\sigma_j(i) = \tilde\tau_j(\tilde\sigma_{j-1}(i)) = (S_i\tau_j)(\tilde\sigma_{j-1}(i)) \,.
    \]
    
    It remains to proof condition \ref{cellular_models:dual_ehrenfried:cob_tr:i} of Definition \ref{cellular_models:dual_ehrenfried:cob_tr}.
    By assumption, $\tilde\Sigma$ is an inner cell, so $\tilde\tau_j = (i\ c)$ for at least one $j$ with $c \neq \tilde\sigma_{j-1}(i)$ by Proposition \ref{cellular_models:ehrenfried:cor_d_hor_deg}.
    We have
    \[
        \tau_j = D_i(\tilde\tau_j (i\ \tilde\sigma_{j-1}(i))) \mspc{hence}{20} S_i\tau_j = (c\ \tilde\sigma_{j-1}(i)) 
    \]
    and therefore
    \[
        (S_i\tau_j)(\tilde\sigma_{j-1}(i)) = c \neq \tilde\sigma_{j-1}(i) \,.
    \]
\end{proof}

The next proposition states in what sense two cofaces might differ.
\begin{prop}
    \label{cellular_models:dual_ehrenfried:difference_of_two_cofaces}
    Let $\Sigma = (\tau_h \mid \ldots \mid \tau_1)$ and $\Sigma' = (\tau_h' \mid \ldots \mid \tau_1')$ be cells of bidegree $(p,h)$ that have their $i\Th$ face in common.
    If we assume $\sigma_j(i) = \sigma'_j(i)$ for all $j$ then $\Sigma = \Sigma'$.
    In any case, the transpositions $\tau_j$ and $\tau_j'$ satisfy
    \begin{enumerate}
        \item[\mylabel{cellular_models:dual_ehrenfried:lem:common_face:i_fix}{(1)}] If $\tau_j(i) = i = \tau_j'(i)$, then
            \[
                \tau_j' = \tau_j \,.
            \]
        \item[(2)]Otherwise we can assume without loss of generality $\tau_j = (i\ c)$.
            \begin{enumerate}
                \item[\mylabel{cellular_models:dual_ehrenfried:lem:common_face:sigmas_equal}{(2.1)}] If in addition $\sigma_{j-1}(i) = \sigma'_{j-1}(i)$, then
                    \[
                        \tau_j' = (\sigma_{j-1}(i)\  c) \mspc{or}{20} \tau_j' = (i\ c) \,. 
                    \]
                \item[\mylabel{cellular_models:dual_ehrenfried:lem:common_face:sigmas_distinct}{(2.2)}] If in addition $\sigma_{j-1}(i) \neq \sigma'_{j-1}(i)$, then
                    \[
                        \tau_j' = (\sigma_{j-1}(i)\  c) \mspc{or}{20} \tau_j' = (i\ \sigma_{j-1}(i)) \,.
                    \]
            \end{enumerate}
    \end{enumerate}
\end{prop}

\begin{proof}
    The first statement is Lemma \ref{cellular_models:dual_ehrenfried:cob_are_equal_via_sigma_j_i}, so we concentrate on the second one.
    We denote the $i\Th$ face of the above cells by $d''_i( \Sigma ) = ( \bar \tau_q \mid \ldots \mid \bar \tau_1 )$ and omit the subscripts since $j$ is fixed.
    Identifying the permutations in $\SymGr_{p-1}$ with their image under the $i\Th$ pseudo degeneracy $S_i \colon \SymGr_{p-1} \xhr{} \SymGr_p$,
    Proposition \ref{cellular_models:parallel:prop_dh} yields
    \begin{align}
        \label{cellular_models:para:lem:common_face:prf_1} & \bar \tau = \tau                          & & \text{for } \tau(i) = i\,, \\
        \label{cellular_models:para:lem:common_face:prf_2} & \bar \tau = ( \sigma(i)\ c ) \neq \id     & & \text{for } \tau = (i\ c)\,,
    \end{align}
    This implies \ref{cellular_models:dual_ehrenfried:lem:common_face:i_fix}.
    The cases \ref{cellular_models:dual_ehrenfried:lem:common_face:sigmas_equal} and \ref{cellular_models:dual_ehrenfried:lem:common_face:sigmas_distinct} follow immediately from
    \[
        ( \sigma(i)\ c ) \overset{\eqref{cellular_models:para:lem:common_face:prf_2}}{=}
            \bar \tau =
            \bar \tau' = 
            \begin{cases}
                \tau' & \text{for } \tau'(i) = i \text{ by \eqref{cellular_models:para:lem:common_face:prf_1}} \\
                ( \sigma'(i)\ c' ) & \text{for } \tau'(i) \neq i \text{ by \eqref{cellular_models:para:lem:common_face:prf_2}}
            \end{cases}\,,
    \]
    since for $( \sigma(i)\ c ) = ( \sigma(i)'\ c' )$, equation $\eqref{cellular_models:para:lem:common_face:prf_2}$ yields
    \[
        \tau' = \begin{cases} (i\ c) & \text{for } \sigma(i) = \sigma'(i) \\ (i\ \sigma(i)) & \text{for } \sigma(i) \neq \sigma'(i) \end{cases}\,.
    \]
\end{proof}
