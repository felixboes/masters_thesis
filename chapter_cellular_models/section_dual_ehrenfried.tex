\section{The Dual Ehrenfried Complex [B]}
\label{cellular_models:dual_ehrenfried}
In this section, we introduce the notion of coboundary traces to provide explicit formulas for both the horizontal coface operator $(\del'')^\ast$ and
the coboundary operator $\del_\E^\ast$.
Moreover, we study some usefull properties of $\kappa^\ast$ and classify the cells of a given Ehrenfried complex.

Let us sketch the process of constructing horizontal cofaces of a given top dimensional cell $\Sigma = \inhom$.
Every transposition defines a slit of pairs where shorter slits sit atop of longer slits if they are of the same height.
In some pictures (e.g.\ Lemma \ref{cellular_models:dual_ehrenfried:f_dual_of_a_cell}), we indicate this pairing by an arc joining the two slits.
In Figure \ref{cellular_models:dual_ehrenfried:motivate_coboundaries_pre} we picture the cell $\Sigma = ((\ul3\ \ul1)|(\ul2\ \ul1))$.
\begin{figure}[ht]
\centering
\incgfx{pictures/cellular_ehrenfried_dual_motivate_coboundaries_pre}
\caption{\label{cellular_models:dual_ehrenfried:motivate_coboundaries_pre}The cell $((\ul3\ \ul1)|(\ul2\ \ul1))$. Here $g=1$, $n=1$ and $m=0$.}
\end{figure}

We obtain a coboundary by glueing a stripe inbetween the two slits of height $\ul 1$.
More sophisticated, we let the shorter slit jump through the longer slit, in order to end up below the long slit of height $\ul2$, and glue in the stripe afterwards.
The two coboundaries are sketched in Figure \ref{cellular_models:dual_ehrenfried:motivate_coboundaries} where we shade the stripe which was glued in.
\begin{figure}[ht]
\centering
\incgfx{pictures/cellular_ehrenfried_dual_motivate_coboundaries}
\caption{\label{cellular_models:dual_ehrenfried:motivate_coboundaries}Two coboundaries of $((\ul3\ \ul1)|(\ul2\ \ul1))$. Still $g=1$, $n=1$ and $m=0$.}
\end{figure}

For a cell with more slits we might have more choices for jumps as seen in the next example.
The cell $\Sigma = \big( (\ul4\ \ul1) | (\ul3\ \ul1) | (\ul3\ \ul1) | (\ul2\ \ul 1) \big)$ is skeched in Figure \ref{cellular_models:dual_ehrenfried:motivate_more_coboundaries_pre}
\begin{figure}[ht]
\centering
\incgfx{pictures/cellular_ehrenfried_dual_motivate_more_coboundaries_pre}
\caption{\label{cellular_models:dual_ehrenfried:motivate_more_coboundaries_pre}The cell $\big( (\ul4\ \ul1) | (\ul3\ \ul1) | (\ul3\ \ul1) | (\ul2\ \ul 1) \big)$. Here $g=2$, $n=1$ and $m=0$.}
\end{figure}
and we construct the three coboundaries seen in Figure \ref{cellular_models:dual_ehrenfried:motivate_more_coboundaries}.
To obtain coboundary (1), we let slits 2, 3 and 4 of height $\ul1$ jump through slit 1.
For coboundary (2), we let slits 3 and 4 of height $\ul1$ jump through slit 2,
and for coboundary (3), we slits 2  and 3 of height $\ul1$ jump through slit 1.
\begin{figure}[ht]
\centering
\incgfx{pictures/cellular_ehrenfried_dual_motivate_more_coboundaries}
\caption{\label{cellular_models:dual_ehrenfried:motivate_more_coboundaries}Three coboundaries of $\big( (\ul4\ \ul1) | (\ul3\ \ul1) | (\ul3\ \ul1) | (\ul2\ \ul 1) \big)$ that arise from different jumps.}
\end{figure}

The key insight is that an $i\Th$ coboundary $\tilde\Sigma$ is determined by its sequence $\tilde\sigma_j(i)$.
By this, we see that constructing a coboundary should be the same as constructing such a sequence by choosing which slits are going to jump.
We encode our geometric intuition in the notion of $i\Th$ coboundary trace (see Definition \ref{cellular_models:dual_ehrenfried:cob_tr})
and show that the canonical map from the set of $i\Th$ coboundary traces $T_i(\Sigma)$ to the set of $i\Th$ cofaces $\cof_i(\Sigma)$ is bijective (see Proposition \ref{cellular_models:dual_ehrenfried:cob_tr_equals_cob}).
The coface corresponding to a given coboundary trace $a$ is denoted by $a.\Sigma$.
Using our notation, the coboundary operator of the Ehrenfried complex is
\[
    \del_\E^\ast(\Sigma) = \sum_{i=1}^p (-1)^i \sum_{a \in T_i(\Sigma)} \kappa^\ast(a.\Sigma) \,.
\]
Most importantly the explicit formula and a better understanding of $\kappa^\ast$ (developed in Subsection \ref{cellular_models:dual_ehrenfried:kappa_dual})
allows us to define homology operations on the moduli space via coboundary maps defined on the dual Ehrenfried complex, see Chapter \ref{homology_operations}.

Moreover, we classify the cells of a fixed Ehrenfried complex.
Cofaces $\tilde\Sigma$ are said to arise as basic expansions of $\Sigma$ if the construction involves no jumps, in the sense of the above paragraphs.
Cells that do not arise this way are called thin.
Applying basic expansions is commutative (this is made precise by Proposition \ref{cellular_models:dual_ehrenfried:basic_expansions_commute})
and every cell in $\E$ arises uniquely from a thin cell by applying a set of basic expansions:
\begin{prop*}[\ref{cellular_models:dual_ehrenfried:every_cell_is_an_expansion}]
    Denote the set of thin cells by $\Thin$, the set of basic expansions of a cell $\Sigma$ by $Bsupp(\Sigma)$ and the power set operator by $\mc Pow$.
    There is a bijection
    \[
        ex \colon \coprod_{\Sigma \in \Thin} \mc Pow( Bsupp(\Sigma)) \to Cells(\E) \mspc{with}{20} Bsupp(\Sigma) \supseteq J \mapsto J.\Sigma \,.
    \]
\end{prop*}
In particular, the number of cells of a given degree is the sum of certain binomial coefficients and the dimension of $\E$ is
\[
    \dim \E = \sum_{p=2}^{2h} 2^{(2h-p)} \cdot |\{\Sigma \in \E_p \text{ thin} \}| \,.
\]

\subsection{The Coface Operator via Coboundary Traces [B]}
In this subsection, we proceed as described above.
We encode our geometric intuition in the notion of $i\Th$ coboundary trace and show
that the canonical map from the set of $i\Th$ coboundary traces $T_i(\Sigma)$ to the set of $i\Th$ cofaces $\cof_i(\Sigma)$ is bijective.
Hereby, we focus on the parallel Ehrenfried complex as our proofs become more compact and the radial case is treated analogously.

Recall that the horizontal boundary operator is defined as the alternating sum of the faces of $\Delta^{p_1} \times \ldots \times \Delta^{p_r}$.
It is therefore computed ``levelwise'' and it might be helpful to think of parallel slit domains with exactly one level.

\begin{defi}
    \label{cellular_models:dual_ehrenfried:cofaces}
    \index{coboundary!coboundary of a cell}
    \index{coboundary!set of coboundaries}
    \symbolindex[c]{$\cof_i(\Sigma)$}{The set of all $i\Th$ cofaces of a given cell $\Sigma$}{Definition \ref{cellular_models:dual_ehrenfried:cofaces}}
    Let $\Sigma = \inhom$ be a non-degenerate cell in the double complex $P/P'$ of bidegree $(p,h)$.
    The set of its $i\Th$ cofaces is denoted by
    \[
        \cof_i(\Sigma) = \{ \tilde\Sigma \in P_{p+1, h} \mid d_i''(\tilde\Sigma) = \Sigma \} \,.
    \]
\end{defi}

\begin{defi}
    \label{cellular_models:dual_ehrenfried:cob_tr}
    \index{coboundary!coboundary trace}
    \index{coboundary!set of coboundary traces}
    \symbolindex[a]{$\homog[a]$}{A coboundary trace of a cell $\Sigma$.}{Definition \ref{cellular_models:dual_ehrenfried:cob_tr}}
    \symbolindex[t]{$T_i(\Sigma)$}{The set of all $i\Th$ coboundary traces of a given cell $\Sigma$.}{Definition \ref{cellular_models:dual_ehrenfried:cob_tr}}
    Let $\Sigma = \inhom$ be a parallel non-degenerate top dimensional cell with respect to $[p]$ and let $\ul 0_k < i \le \ul p_k$.
    A sequence $\homog[a]$ in $[p]$ is called $i\Th$ {\bf coboundary trace of $\Sigma$} if it satisfies the following conditions:
    \begin{enumerate}
        \item \label{cellular_models:dual_ehrenfried:cob_tr:normalization} $a_0 = i+1$.
        \item \label{cellular_models:dual_ehrenfried:cob_tr:choice} If $a_j \neq a_{j-1}$, then $a_j = (S_i\tau_j)(a_{j-1})$
                    (or equivalently: if $a_j \neq (S_i\tau_j)(a_{j-1})$, then $a_j = a_{j-1}$).
        \item \label{cellular_models:dual_ehrenfried:cob_tr:i} $a_j \neq (S_i\tau_j)(a_{j-1})$ at least once.
        \item \label{cellular_models:dual_ehrenfried:cob_tr:i+1} $a_j \neq a_{j-1}$ at least once.
    \end{enumerate}
    The set of all $i\Th$ coboundary traces of $\Sigma$ is denoted by
    \[
        T_i(\Sigma) = \{ (a_h : \ldots : a_0) \text{ is an $i\Th$ coboundary trace of $\Sigma$} \} \,.
    \]
\end{defi}

Let us elaborate on the above definition.
Condition \ref{cellular_models:dual_ehrenfried:cob_tr:normalization} is the normalization corresponding to $\sigma_0(i) = i+1$.
The second condition encodes jumping slits and glueing in stripes.
In this sense, condition \ref{cellular_models:dual_ehrenfried:cob_tr:i} forbids glueing in a stripe below all stripes of height $i$ and
condition \ref{cellular_models:dual_ehrenfried:cob_tr:i+1} forbids glueing in a stripe above all stripes of height $i$.
Recalling that the $\ul0_k\Th$ slit of a radial cells might be empty the next definition is the obvious analogue to Definition \ref{cellular_models:dual_ehrenfried:cob_tr}.

\begin{defi}
    \label{cellular_models:dual_ehrenfried:cob_tr_radial}
    Let $\Sigma = \inhom$ be a radial non-degenerate top dimensional cell with respect to $[p]$ and let $\ul 0 < i \le \ul{p+1}_k$.
    A sequence $\homog[a]$ in $[p]$ is called $i\Th$ {\bf coboundary trace of $\Sigma$} if it satisfies the following conditions:
    \begin{enumerate}
        \item \label{cellular_models:dual_ehrenfried:cob_tr_radial:normalization} $a_0 = \begin{cases}i+1& i \neq \ul{p+1}_k \\ \ul0_k & i = \ul {p+1}_k\end{cases}$.
        \item \label{cellular_models:dual_ehrenfried:cob_tr_radial:choice} If $a_j \neq a_{j-1}$, then $a_j = (S_i\tau_j)(a_{j-1})$
                    (or equivalently: if $a_j \neq (S_i\tau_j)(a_{j-1})$, then $a_j = a_{j-1}$).
        \item \label{cellular_models:dual_ehrenfried:cob_tr_radial:i} If $i \neq \ul0_k$ we have $a_j \neq (S_i\tau_j)(a_{j-1})$ at least once.
        \item \label{cellular_models:dual_ehrenfried:cob_tr_radial:i+1} If $i \neq \ul{p+1}_k$ we have $a_j \neq a_{j-1}$ at least once.
    \end{enumerate}
    The set of all $i\Th$ coboundary traces of $\Sigma$ is denoted by
    \[
        T_i(\Sigma) = \{ (a_h : \ldots : a_0) \text{ is an $i\Th$ coboundary trace of $\Sigma$} \} \,.
    \]    
\end{defi}


\begin{rem}
    \label{cellular_models:dual_ehrenfried:cob_tr_do_not_contain_i}
    Observe that the symbol $i$ does not occur in an $i\Th$ coboundary trace.
\end{rem}

\begin{defi}
    \label{cellular_models:dual_ehrenfried:a_dot_sigma}
    \symbolindex[a]{$a.\Sigma$}{The coboundary of $\Sigma$ corresponding to $a$.}{Definition \ref{cellular_models:dual_ehrenfried:a_dot_sigma}}
    Let $\Sigma = \inhom$ be a non-degenerate top dimensional cell with respect to $[p]$ and let $i = \ul i_k \in [p]$ and $a \in T_i(\Sigma)$.
    Then we define
    \[
        a.\Sigma = \inhom[\tilde\tau]
    \]
    with
    \[
        \tilde\tau_j =
            \begin{cases}
                S_i\tau_j                               & a_j = (S_i\tau_j)(a_{j-1}) \\
                (i\ a_{j-1}) S_i\tau_j (i\ a_{j-1})     & a_j \neq (S_i\tau_j)(a_{j-1})
            \end{cases}
    \]
    and with respect to the partition of $p+1 = p_1 + \ldots + p_{k-1} + (p_k + 1 ) + p_{k+1} + \ldots + p_r$.
\end{defi}

\begin{rem}
    \label{cellular_models:dual_ehrenfried:tilde_tau_i_iS_i}
    For $a \in T_i(\Sigma)$ and $a.\Sigma = \inhom[\tilde\tau]$, we have
    \[
        a_j \neq (S_i\tau_j)(a_{j-1}) \mspc{iff}{10} \tilde\tau_j(i) \neq i \mspc{iff}{10} \tilde\tau_j = (i\ (S_i\tau_j)(a_{j-1})) \,.
    \]
\end{rem}

\begin{prop}
    \label{cellular_models:dual_ehrenfried:cob_tr_equals_cob}
    Consider $P=P(h,m;r_1, \ldots, r_n)$, the bisimplicial complex associated with $\Modspc$.
    For every non-degenerate cell $\Sigma \in P_{p,h}$ and $\ul1_k \le i \le \ul p_k$, the map
    \[
        \Phi \colon T_i(\Sigma) \to \cof_i(\Sigma) \mspc{with}{20} a \mapsto a.\Sigma
    \]
    is bijective.
    The $p\Th$ coboundary operator of the associated Ehrenfried complex is therefore
    \[
        \del_\E^\ast( \Sigma ) = \sum_{i=1}^p (-1)^i \sum_{a \in T_i(\Sigma)} \kappa^\ast(a.\Sigma) \,.
    \]
\end{prop}

We prove this proposition using the following basic properties.

\begin{lem}
    \label{cellular_models:dual_ehrenfried:cob_are_equal_via_sigma_j_i}
    Consider $\Sigma = (\tau_h \mid \ldots \mid \tau_1)$ and $\Sigma' = (\tau_h' \mid \ldots \mid \tau_1')$ of bidegree $(p,h)$
    that have their $i\Th$ horizontal face in common.
    Assume that $\sigma_j(i) = \sigma'_j(i)$ for all $j$.
    Then already $\Sigma = \Sigma'$.
\end{lem}

\begin{proof}
    Using the definition of the horizontal differential (see \ref{cellular_models:parallel:faces}), we have
    \[
        (i\ \sigma_j(i)) \sigma_j = D_i(\sigma_j) = D_i(\sigma'_j) = (i\ \sigma'_j(i)) \sigma'_j = (i\ \sigma_j(i)) \sigma'_j
    \]
    for arbitrary $j$, up to renormalization.
    Hence $\sigma_j = \sigma'_j$ for all $j$ and the claim follows.
\end{proof}

\begin{lem}
    \label{cellular_models:dual_ehrenfried:cob_tr_def_sigma_j_i}
    Let $a = \homog[a]$ be an $i\Th$ coboundary trace of $\Sigma$ and denote $a.\Sigma = \homog[\tilde\sigma]$.
    Then we have
    \[
        a_j = \tilde\sigma_j(i) \,.
    \]
\end{lem}

\begin{proof}
    By construction $a_0 = i+1 = \tilde\sigma_0(i)$.
    We assume there is a minimal index $j$ with $a_j \neq \tilde\sigma_j(i)$.
    Hence
    \[
        a_j \neq \tilde\sigma_j(i) =
            \tilde\tau_j(\tilde\sigma_{j-1}(i)) =
            \tilde\tau_j(a_{j-1}) =
            \begin{cases}
                (S_i\tau_j)(a_{j-1})                                    & a_j = (S_i\tau_j)(a_{j-1}) \\
                \big( (i\ a_{j-1}) S_i\tau_j (i\ a_{j-1}) \big)(a_{j-1}) = a_{j-1}  & a_j \neq (S_i\tau_j)(a_{j-1})
            \end{cases}
    \]
    by definition of $\tilde\tau_j$.
    The first case is clearly impossible, and the second case implies $\sigma_j(i) = a_{j-1} = a_j$ by \ref{cellular_models:dual_ehrenfried:cob_tr:choice} in Definition \ref{cellular_models:dual_ehrenfried:cob_tr}.
\end{proof}

\begin{lem}
    \label{cellular_models:dual_ehrenfried:a_Sigma_is_cob}
    Let $a \in T_i(\Sigma)$.
    Then
    \[
        d_i''(a.\Sigma) = \Sigma \,.
    \]
\end{lem}

\begin{proof}
    Denote $a.\Sigma = \inhom[\tilde\tau] = \homog[\tilde\sigma]$ and $d_i''(a.\Sigma) = \inhom[\tilde\tau'']$.
    For $q \ge j \ge 1$, by Proposition \ref{cellular_models:parallel:prop_dh} and Lemma \ref{cellular_models:dual_ehrenfried:cob_tr_def_sigma_j_i},
    we have
    \[
        \tilde\tau_j'' = D_i( \tilde\tau_j \cdot (i\ a_{j-1})) = 
            \begin{cases}
                D_i( S_i\tau_j \cdot (i\ a_{j-1})) & a_j = (S_i\tau_j)(a_{j-1}) \\
                D_i( (i\ a_{j-1}) \cdot S_i\tau_j) & a_j \neq (S_i\tau_j)(a_{j-1}) \\
            \end{cases} \,.
    \]
    Now $(i\ a_{j-1})$ is disregarded in both cases by Proposition \ref{cellular_models:parallel:prop_dh} and we are done as $D_iS_i\tau_j = \tau_j$.
\end{proof}

\begin{lem}
    \label{cellular_models:dual_ehrenfried:a_Sigma_is_non_deg}
    Consider a non-degenerate cell $\Sigma \in P_{p,h}$ and let $a \in T_i(\Sigma)$.
    Then the cell $a.\Sigma \in P_{p+1,h}$ is also non-degenerate.
\end{lem}

\begin{proof}
    Denote
    \[
        \Sigma = \inhom = \homog \mspc{and}{20} a.\Sigma = \tilde\Sigma = \inhom[\tilde\tau] = \homog[\tilde\sigma] \,.
    \]
    We show that $\tilde\Sigma$ is a connected inner cell with the correct number of punctures and boundaries.
    
    Clearly, $\tilde\tau_j \neq 1$ for all $q \ge j \ge 1$.
    Recall that the $\tilde\tau_q, \ldots, \tilde\tau_1$ have a fixed point $k$ in common if and only if $\tilde\sigma_j(k-1) = k$ for all $j$.
    By assumtion, $D_i(\tilde\sigma_j) = \sigma_j$ and $\Sigma$ is an inner cell.
    Hence, it suffices to check $k=i,i+1$.
    By condition \ref{cellular_models:dual_ehrenfried:cob_tr:i} in Definition \ref{cellular_models:dual_ehrenfried:cob_tr},
    there is at least one $a_j \neq (S_i\tau_j)(a_{j-1})$.
    This implies $\tilde\tau_j(i) \neq i$ (see Remark \ref{cellular_models:dual_ehrenfried:tilde_tau_i_iS_i}).
    By condition \ref{cellular_models:dual_ehrenfried:cob_tr:i+1} in Definition \ref{cellular_models:dual_ehrenfried:cob_tr},
    there is at least one $j$ with $a_j \neq a_{j-1}$.
    This implies $\tilde\sigma_j(i) \neq i+1$ for at least one $j$.
    Thus $\tilde\Sigma$ is an inner cell.
    
    By Lemma \ref{cellular_models:dual_ehrenfried:a_Sigma_is_cob}, $\tilde\Sigma$ is an $i\Th$ coboundary of $\Sigma$.
    In particular, it is connected.
    Moreover, $\tilde\sigma_j(i) = a_j \neq i$ (by Lemma \ref{cellular_models:dual_ehrenfried:cob_tr_def_sigma_j_i} and Remark \ref{cellular_models:dual_ehrenfried:cob_tr_do_not_contain_i}) 
    and $D_i(\tilde\sigma_j) = \sigma_j$.
    Therefore
    \[
        \ncyc(\tilde\Sigma) = \ncyc(\Sigma)
    \]
    and, by construction, $N(\tilde\Sigma) = h$.
    The levels of $\tilde\Sigma$ are ordered ascendingly as this is true for $\Sigma$.
\end{proof}

\begin{proof}[Proof of Proposition \ref{cellular_models:dual_ehrenfried:cob_tr_equals_cob}]
    The map $\Phi$ is well defined by Lemma \ref{cellular_models:dual_ehrenfried:a_Sigma_is_non_deg} and Lemma \ref{cellular_models:dual_ehrenfried:a_Sigma_is_cob}.
    
    By Lemma \ref{cellular_models:dual_ehrenfried:cob_tr_def_sigma_j_i},
    every $i\Th$ coboundary trace $a\in T_i(\Sigma)$ defines a coboundary $a.\Sigma = \homog[\tilde\sigma]$ with $a_j = \tilde\sigma_j(i)$.
    We conclude that $\Phi$ is injective (as a consequence of Lemma \ref{cellular_models:dual_ehrenfried:cob_are_equal_via_sigma_j_i}).
    
    Using Lemma \ref{cellular_models:dual_ehrenfried:cob_tr_def_sigma_j_i} and Lemma \ref{cellular_models:dual_ehrenfried:cob_are_equal_via_sigma_j_i},
    it remains to show that every $i\Th$ coboundary $\tilde\Sigma = \homog[\tilde\sigma]$ defines an $i\Th$ coboundary trace $a$ of $\Sigma$ with $a_j = \tilde\sigma_j(i)$.
    Both the conditions
    \[
        \tilde\sigma_0(i) = i+1 \mspc{and}{20} \tilde\sigma_j(i) \neq \tilde\sigma_{j-1}(i) \text{ at least once}
    \]
    are clearly satisfied; it remains to prove \ref{cellular_models:dual_ehrenfried:cob_tr:choice} and \ref{cellular_models:dual_ehrenfried:cob_tr:i}.
    
    In order to show condition \ref{cellular_models:dual_ehrenfried:cob_tr:choice} of Definition \ref{cellular_models:dual_ehrenfried:cob_tr},
    let $\tilde\sigma_{j-1}(i) \neq \tilde\sigma_j(i) = \tilde\tau_j(\tilde\sigma_{j-1}(i))$.
    But $\tilde\sigma_{j-1}(i) \neq i \neq \tilde\sigma_j$ by Proposition \ref{cellular_models:ehrenfried:cor_d_hor_deg}, hence $\tau_j(i) = i$.
    Now, by Proposition \ref{cellular_models:parallel:prop_dh},
    \[
        \tau_j = D_i(\tilde\tau_j (i\ \tilde\sigma_{j-1}(i))) = D_i( \tilde\tau_j)\,, \mspc{hence}{20} \tilde\tau_j = S_i\tau_j \,.
    \]
    We have shown that
    \[
        \tilde\sigma_j(i) \neq \tilde\sigma_{j-1}(i) \mspc{implies}{20} \tilde\sigma_j(i) = \tilde\tau_j(\tilde\sigma_{j-1}(i)) = (S_i\tau_j)(\tilde\sigma_{j-1}(i)) \,.
    \]
    
    It remains to proof condition \ref{cellular_models:dual_ehrenfried:cob_tr:i} of Definition \ref{cellular_models:dual_ehrenfried:cob_tr}.
    By assumption, $\tilde\Sigma$ is an inner cell, so $\tilde\tau_j = (i\ c)$ for at least one $j$ with $c \neq \tilde\sigma_{j-1}(i)$ by Proposition \ref{cellular_models:ehrenfried:cor_d_hor_deg}.
    We have
    \[
        \tau_j = D_i(\tilde\tau_j (i\ \tilde\sigma_{j-1}(i))) \mspc{hence}{20} S_i\tau_j = (c\ \tilde\sigma_{j-1}(i)) 
    \]
    and therefore
    \[
        (S_i\tau_j)(\tilde\sigma_{j-1}(i)) = c \neq \tilde\sigma_{j-1}(i) \,.
    \]
\end{proof}

The next proposition states in what sense two cofaces might differ.
\begin{prop}
    \label{cellular_models:dual_ehrenfried:difference_of_two_cofaces}
    Let $\Sigma = (\tau_h \mid \ldots \mid \tau_1)$ and $\Sigma' = (\tau_h' \mid \ldots \mid \tau_1')$ be cells of bidegree $(p,h)$ that have their $i\Th$ face in common.
    If we assume $\sigma_j(i) = \sigma'_j(i)$ for all $j$ then $\Sigma = \Sigma'$.
    In any case, the transpositions $\tau_j$ and $\tau_j'$ satisfy
    \begin{enumerate}
        \item[\mylabel{cellular_models:dual_ehrenfried:lem:common_face:i_fix}{(1)}] If $\tau_j(i) = i = \tau_j'(i)$, then
            \[
                \tau_j' = \tau_j \,.
            \]
        \item[(2)]Otherwise we can assume without loss of generality $\tau_j = (i\ c)$.
            \begin{enumerate}
                \item[\mylabel{cellular_models:dual_ehrenfried:lem:common_face:sigmas_equal}{(2.1)}] If in addition $\sigma_{j-1}(i) = \sigma'_{j-1}(i)$, then
                    \[
                        \tau_j' = (\sigma_{j-1}(i)\  c) \mspc{or}{20} \tau_j' = (i\ c) \,. 
                    \]
                \item[\mylabel{cellular_models:dual_ehrenfried:lem:common_face:sigmas_distinct}{(2.2)}] If in addition $\sigma_{j-1}(i) \neq \sigma'_{j-1}(i)$, then
                    \[
                        \tau_j' = (\sigma_{j-1}(i)\  c) \mspc{or}{20} \tau_j' = (i\ \sigma_{j-1}(i)) \,.
                    \]
            \end{enumerate}
    \end{enumerate}
\end{prop}

\begin{proof}
    The first statement is Lemma \ref{cellular_models:dual_ehrenfried:cob_are_equal_via_sigma_j_i}, so we concentrate on the second one.
    We denote the $i\Th$ face of the above cells by $d''_i( \Sigma ) = ( \bar \tau_q \mid \ldots \mid \bar \tau_1 )$ and omit the subscripts since $j$ is fixed.
    Identifying the permutations in $\SymGr_{p-1}$ with their image under the $i\Th$ pseudo degeneracy $S_i \colon \SymGr_{p-1} \xhr{} \SymGr_p$,
    Proposition \ref{cellular_models:parallel:prop_dh} yields
    \begin{align}
        \label{cellular_models:para:lem:common_face:prf_1} & \bar \tau = \tau                          & & \text{for } \tau(i) = i\,, \\
        \label{cellular_models:para:lem:common_face:prf_2} & \bar \tau = ( \sigma(i)\ c ) \neq \id     & & \text{for } \tau = (i\ c)\,,
    \end{align}
    This implies \ref{cellular_models:dual_ehrenfried:lem:common_face:i_fix}.
    The cases \ref{cellular_models:dual_ehrenfried:lem:common_face:sigmas_equal} and \ref{cellular_models:dual_ehrenfried:lem:common_face:sigmas_distinct} follow immediately from
    \[
        ( \sigma(i)\ c ) \overset{\eqref{cellular_models:para:lem:common_face:prf_2}}{=}
            \bar \tau =
            \bar \tau' = 
            \begin{cases}
                \tau' & \text{for } \tau'(i) = i \text{ by \eqref{cellular_models:para:lem:common_face:prf_1}} \\
                ( \sigma'(i)\ c' ) & \text{for } \tau'(i) \neq i \text{ by \eqref{cellular_models:para:lem:common_face:prf_2}}
            \end{cases}\,,
    \]
    since for $( \sigma(i)\ c ) = ( \sigma(i)'\ c' )$, equation $\eqref{cellular_models:para:lem:common_face:prf_2}$ yields
    \[
        \tau' = \begin{cases} (i\ c) & \text{for } \sigma(i) = \sigma'(i) \\ (i\ \sigma(i)) & \text{for } \sigma(i) \neq \sigma'(i) \end{cases}\,.
    \]
\end{proof}

\subsection{The Dual of \texorpdfstring{$\kappa$}{kappa}}
\label{cellular_models:dual_ehrenfried:kappa_dual}
\symbolindex[k]{$\kappa^\ast$}{The dual of $\kappa$}{Subsection \ref{cellular_models:dual_ehrenfried:kappa_dual}}
The map $\pi^\ast$ is the canonical inclusion and we understood the horizontal coboundary operator $(\del'')^\ast$ in terms of coboundary traces via Proposition \ref{cellular_models:dual_ehrenfried:cob_tr_equals_cob}.
It remains to gain some insights on the dual of $\kappa$.

\begin{defi}
    \label{cellular_models:dual_ehrenfried:kappa_dual_sequences}
    \index{kappa!kappa star sequence}
    \symbolindex[k]{$\kappa^\ast_J$}{The summand $\kappa^\ast_J = \mueta^\ast_{j_1} \circ \ldots \circ \mueta^\ast_{j_k}$ with $J = (j_1, \ldots, j_k)$.}{Definition \ref{cellular_models:dual_ehrenfried:kappa_dual_sequences}}
    \symbolindex[1]{$\mueta^\ast$ or $\mueta^\ast_j$}{The dual of $\mueta$ or $\mueta_j$}{Definition \ref{cellular_models:dual_ehrenfried:kappa_dual_sequences}}
    \symbolindex[l]{$\Lambda^\ast_h$}{Parametrizes the $\kappa^\ast$-squences of length $h$.}{Definition \ref{cellular_models:dual_ehrenfried:kappa_dual_sequences}}
    Let $a$ and $b$ be positive integers.
    Denote $J_a^b = (b,b-1, \ldots, a+1, a)$ for $a \le b$ and let $J_a^b = ()$ be the empty sequence for $a > b$.
    The set of {\bfseries $\kappa^\ast$-sequences} is
    \[
        \Lambda^\ast_1 \{ () \} \mspc{and by concatenation}{20} \Lambda^\ast_{n+1} = \Lambda^\ast_n.\{ J_1^n, \ldots, J_n^n, J_{n+1}^n \} \,.
    \]
    For a $\kappa^\ast$-sequence $J = (i_1, \ldots, i_k)$ we set
    \[
        \kappa^\ast_J = \mueta^\ast_{i_1} \circ \ldots \circ \mueta^\ast_{i_k} \,.
    \]
\end{defi}

\begin{lem}
    \label{cellular_models:dual_ehrenfried:formula_for_kappa_dual}
    The map $\kappa^\ast$ is the alternating sum of all $\kappa^\ast$-sequences:
    \[
        \kappa^\ast_h = \sum_{(i_1, \ldots, i_k) \in \Lambda^\ast_h} (-1)^k \kappa^\ast_{(i_1, \ldots, i_k)} \,.
    \]
\end{lem}

\begin{proof}
    This is just the dual statement of Lemma \ref{cellular_models:ehrenfried:formula_for_kappa}.
\end{proof}

To get our hands on $\kappa^\ast$, we have to understand $\mueta_j^\ast$.
From the definition of the factorization map and $\kappa^\ast$,
it suffices to examine the image of a cell $(\tau_2 \mid \tau_1)$ of bidegree $(p,2)$ with $p = 2,3,4$ under $\mueta = \mueta_1$.

\begin{lem}
    \label{cellular_models:dual_ehrenfried:f_dual_of_a_cell}
    We have
    \begin{align}
        \tag{1.1} \mueta^\ast \left( \vierzweizelle{2}{1}{4}{3} \right) & = \vierzweizelle{2}{1}{4}{3} + \vierzweizelle{4}{3}{2}{1} \\
        \tag{1.2} \mueta^\ast \left( \vierzweizelle{3}{1}{4}{2} \right) & = \vierzweizelle{3}{1}{4}{2} + \vierzweizelle{4}{2}{3}{1} \\
        \tag{1.3} \mueta^\ast \left( \vierzweizelle{3}{2}{4}{1} \right) & = \vierzweizelle{3}{2}{4}{1} + \vierzweizelle{4}{1}{3}{2} \\
        \tag{2.1} \mueta^\ast \left( \dreizweizelle{2}{1}{3}{2} \right) & = \dreizweizelle{2}{1}{3}{2} + \dreizweizelle{3}{2}{3}{1} + \dreizweizelle{3}{1}{2}{1} \\
        \tag{2.2} \mueta^\ast \left( \dreizweizelle{2}{1}{3}{1} \right) & = \dreizweizelle{2}{1}{3}{1} + \dreizweizelle{3}{2}{2}{1} + \dreizweizelle{3}{1}{3}{2} \\
        \intertext{and for every $\Sigma$ not listed above we have}
        \tag{3}   \mueta^\ast (\Sigma) &= 0
    \end{align}
\end{lem}

\begin{proof}
    This is follows directly from a case-by-case analysis of $\mueta(\tau_2 \mid \tau_1)$ for all inner cells of bidegree $(p,2)$ with $p = 2,3,4$.
\end{proof}

\begin{lem}
    \label{cellular_models:dual_ehrenfried:f_dual_vanishes_at_monotonous_spot}
    Let $\Sigma = \inhomq$ be an inner cell with $\hgt(\tau_{j+1}) > \hgt(\tau_j)$ for some $q > j \ge 1$.
    Then
    \[
        \mueta^\ast_j(\Sigma) = 0 \,.
    \]
\end{lem}

\begin{proof}
    This follows immediately from the definition of the factorization map or from the lemma above.
\end{proof}

\begin{defi}
    \label{cellular_models:dual_ehrenfried:relevant_kappa_dual_sequences}
    \index{kappa!relevant kappa star sequence}
    \index{kappa!irrelevent kappa star sequence}
    \symbolindex[r]{$R^\Sigma$}{The set of relevant $\kappa^\ast$-sequences.}{Definition \ref{cellular_models:dual_ehrenfried:relevant_kappa_dual_sequences}}
    Let $\Sigma$ be a top dimensional cell.
    A $\kappa^\ast$-sequence $I \in \Lambda^\ast$ is {\bfseries relevant} if $\kappa^\ast_I(\Sigma) \neq 0$ and {\bfseries irrelevant} else.
    The set of relevant $\kappa^\ast$-sequences with respect to $\Sigma$ is denoted by $R^\Sigma$.
\end{defi}

The next lemma will become handy in the study of homology operations see Chapter \ref{homology_operations}.
It states that every relevant $\kappa^\ast$-sequences of a cell pictured in Figure \ref{cellular_models:dual_ehrenfried:motivate_lemma_for_multiplication}
is (up to a canonical shift) the concaternation of $\kappa^\ast$-sequences of $\Sigma'$ and $\Sigma''$
\begin{figure}[ht]
\centering
\incgfx{pictures/cellular_ehrenfried_dual_motivate_lemma_for_multiplication}
\caption{\label{cellular_models:dual_ehrenfried:motivate_lemma_for_multiplication}This cell might be seen as the product of $\Sigma'$ and $\Sigma''$ (see Definition \ref{homology_operations:parallel_patching_slit_pics:mu}).}
\end{figure}

\begin{lem}
    \label{cellular_models:dual_ehrenfried:relevant_kappa_dual_sequences_of_blocks}
    Consider a top dimensional cell $\Sigma = (\tau_{t+q} \mid \ldots \mid \tau_{t+1} \mid \tau_t \mid \ldots \mid \tau_1)$ with
    \[
        \supp(\tau_t, \ldots, \tau_1) \subseteq \{ \ul 1, \ldots, \ul s \} \mspc{and}{20} \supp(\tau_{t+q}, \ldots, \tau_{t+1}) \subseteq \{ \ul{s+1}, \ldots, \ul{s+p} \} \,.
    \]
    Then, the set of relevant $\kappa^\ast$-sequences with respect to $\Sigma$ is the concatenation
    \[
        R^\Sigma = R^{(\tau_t \mid \ldots \mid \tau_1)}.S_tR^{(\tau_{t+q} \mid \ldots \mid \tau_{t+1})} \,,
    \]
    where $S_t$ is defined on $\kappa^\ast$-sequences to be $S_t( i_1, \ldots, i_k ) = (t+i_1, \ldots, t+i_k)$.
\end{lem}

\begin{proof}
    The supports of $\tau_{t+q}, \ldots, \tau_{t+1}$ and $\tau_t, \ldots, \tau_1$ satisfy the inequality
    \[
         \min\supp(\tau_{t+q}, \ldots, \tau_{t+1})  > \max\supp(\tau_t, \ldots, \tau_1)
    \]
    and so does every term of $\kappa^\ast_I (\Sigma)$ for $I$ a relevant $\kappa^\ast$-sequence (compare Lemma \ref{cellular_models:dual_ehrenfried:f_dual_of_a_cell}).
    Then, by Lemma \ref{cellular_models:dual_ehrenfried:f_dual_vanishes_at_monotonous_spot}, there is not a single $\kappa^\ast$-sequence $(i_1, \ldots, i_k)$ with $i_j = t$ for some $j$.
    Therefore every relevant $\kappa^\ast$-sequence $I \in R^\Sigma$ is the concatenation of two relevant $\kappa^\ast$-sequences $I \in R^{(\tau_t \mid \ldots \mid \tau_1)}.S_tR^{(\tau_{t+q} \mid \ldots \mid \tau_{t+1})}$ and vice versa.
\end{proof}

\subsection{Classification of the Cells of the Ehrenfried Complex}
\label{cellular_models:dual_ehrenfried:classification_of_the_cells}
In this subsection, we encode the geometric ideas presented in the first paragraphs of Section \ref{cellular_models:dual_ehrenfried}
in order to study cofaces that are obtained by glueing a stripe inbetween two slits of the same height.
As we concentrate on the Ehrenfried complex $\E$ (and its dual), we are only interested in top dimensional cells of the bicomplex.
Hence the position of a stripe, which is about to be glued in, is just a coordinate $(j,i)$ with $h \ge j \ge 1$ and $p \ge i \ge 1$.
Proposition \ref{cellular_models:dual_ehrenfried:basic_expansions_commute} states that glueing in different stripes is commutative (up to relabeling the heights).
A cell that does not arise from such a process will be called thin and Proposition \ref{cellular_models:dual_ehrenfried:every_cell_is_an_expansion}
states that every cell of $\E$ is uniquely obtained from a thin cell by such an expansion.

\begin{defi}
    \label{cellular_models:dual_ehrenfried:basic_coboundary_traces}
    \index{coboundary!basic coboundary trace}
    \index{cell!basic expansion of a cell}
    Consider a cell $\Sigma \in \E$.
    An $i\Th$ coboundary trace $\inhom[a]$ is {\bfseries basic} it there exists an index $j$ with
    \begin{enumerate}
        \item $a_{j-1} = \ldots = a_0 = i+1$,
        \item $a_j = (S_i\tau_j)(a_{j-1}) \neq a_{j-1}$ (i.e.\ $\tau_i = (i\ d^\Delta_i(a_j))$) and
        \item $a_{k+1} = (S_i\tau_{k+1})(a_k)$ for $h \ge k \ge j$.
    \end{enumerate}
    In this case, the coface $a.\Sigma$ is called {\bfseries basic expansion} of $\Sigma$.
\end{defi}

\begin{lem}
    \label{cellular_models:dual_ehrenfried:basic_coboundary_traces_have_unique_j_and_i}
    Let $a = \homog[a]$ be a basic coboundary trace of $\Sigma$.
    Then, the $j$ mentioned in Definition \ref{cellular_models:dual_ehrenfried:basic_coboundary_traces} is unique and
    $\homog[a]$ is an $i\Th$ coboundary trace with $i = \tau_j( d_i^\Delta(a_j) )$.
    Moreover,
    \begin{align}
        \label{cellular_models:dual_ehrenfried:basic_coboundary_trace_applied_to_cell}
        a.\Sigma = (S_i\tau_h \mid \ldots \mid S_i\tau_j \mid S_{i+1}\tau_{j-1} \mid \ldots \mid S_{i+1}\tau_1) \,.
    \end{align}

\end{lem}

\begin{proof}
    The index $j$ is clearly unique and $a_j$ fulfills $a_j = (S_i\tau_j)(i+1) \neq i+1$, so $a_j = s_i^\Delta( \tau_j(i) )$ or equivalently $d_i^\Delta(a_j) = \tau_j(i)$.
    
    Equation \eqref{cellular_models:dual_ehrenfried:basic_coboundary_trace_applied_to_cell} is readily verified using Definitions
    \ref{cellular_models:dual_ehrenfried:a_dot_sigma} and \ref{cellular_models:dual_ehrenfried:basic_coboundary_traces}.
\end{proof}

\begin{lem}
    \label{cellular_models:dual_ehrenfried:Btrace_is_Bsupp}
    \symbolindex[b]{$Btrace(\Sigma)$}{The set of basic coboundary traces of $\Sigma$.}{Lemma \ref{cellular_models:dual_ehrenfried:Btrace_is_Bsupp}}
    \symbolindex[b]{$Bsupp(\Sigma)$}{This set is canonically identified with the set of basic coboundary traces of $\Sigma$.}{Lemma \ref{cellular_models:dual_ehrenfried:Btrace_is_Bsupp}}
    For $\Sigma \in \E_p$, the set $Btrace(\Sigma)$ of basic coboundary traces is in one-to-one correspondence to the disjoint union
    \[
        Bsupp(\Sigma) = \coprod_{h \ge j \ge 1} \supp(\tau_j) \cap \supp(\tau_{j-1}, \ldots, \tau_1) \,,
    \]
    where $\homog[a] \in Btrace(\Sigma)$ is mapped to the unique index $i$ in the $j\Th$ component,
    with $j$ and $i = \tau_j( d_i^\Delta(a_j) )$ as in Lemma \ref{cellular_models:dual_ehrenfried:basic_coboundary_traces_have_unique_j_and_i}.
    In particular, the number of basic coboundary traces is
    \[
        |Btrace(\Sigma)| = |Bsupp(\Sigma)| = 2h-p \,.
    \]
\end{lem}

\begin{proof}
    By Definitions \ref{cellular_models:dual_ehrenfried:cob_tr} and \ref{cellular_models:dual_ehrenfried:basic_coboundary_traces},
    the sequence $\homog[a]$ is a basic coboundary trace with respect to $j$ if and only if
    \begin{enumerate}
        \item $a_{j-1} = \ldots = a_0 = i+1$,
        \item $a_k \neq (S_i\tau_k)(a_{k-1})$ at least once,
        \item $a_j = (S_i\tau_j)(a_{j-1})$ and
        \item $a_k = (S_i\tau_k)(a_{k-1})$ for $h \ge k > j$.
    \end{enumerate}
    Thus, the indicated map is a bijection $Btrace(\Sigma) \cong Bsupp(\Sigma)$.

    A symbol $i$ occures in $Bsupp(\Sigma)$ exactly $k$ times if and only if it is in the support of exactly $k+1$ transpositions.
    Thus
    \begin{align*}
        \left| \coprod_{h \ge j \ge 1} \supp(\tau_j) \cap \supp(\tau_{j-1}, \ldots, \tau_1) \right| 
            &= \sum_i \left( \left( \sum_j \left| \{i \} \cap \supp(\tau_j) \right| \right) - 1 \right)\\
            &= \left( \sum_{i,j} \left| \{i \} \cap \supp(\tau_j) \right| \right) - p \\
            &= \left( \sum_j \left| \supp(\tau_j) \right| \right) - p \\
            &= 2h-p \,.
    \end{align*}
\end{proof}

\begin{notation}
    \label{cellular_models:dual_ehrenfried:j_notation}%
    \symbolindex[j]{$j^\epsilon$}{In order to classify the cells of the Ehrenfried complex, we need a more handy notation for basic coboundary traces.}{Notation \ref{cellular_models:dual_ehrenfried:j_notation}} 
    In order to forumlate Proposition \ref{cellular_models:dual_ehrenfried:basic_expansions_commute} we introduce yet another notation.
    We want to ignore the index shifts that occures if we compare $Bsupp(\Sigma)$ with $Bsupp(a.\Sigma)$ for a basic coboundary trace $a = \homog[a]$:
    It suffices to compare the relative index in the support of every transposition.
    For $s_j = \supp(\tau_j) \cap \supp(\tau_{j-1}, \ldots, \tau_1)$, we have $|s_j| \le 2$.
    Thus we write
    \[
        s_j \ni c = j^\eps \mspc{with}{20} \eps = \begin{cases} 0 & c = \min(s_j) \\ 1 & c = \max(s_j) \end{cases} \mspc{and identify}{20} j^0 = j^1 \mspc{if}{10} |s_j| = 1 \,.
    \]
    Using the bijection in Lemma \ref{cellular_models:dual_ehrenfried:Btrace_is_Bsupp}, we write $a(j^\eps)$ for the basic coboundary trace corresponding to $j^\eps \in Bsupp(\Sigma)$ and
    denote by $j^\eps.\Sigma$ the coboundary $a(j^\eps).\Sigma$.
\end{notation}

\begin{prop}
    \label{cellular_models:dual_ehrenfried:basic_expansions_commute}
    Using the above notation, let $j^\eps \in Bsupp(\Sigma)$.
    Then
    \begin{align}
        \label{cellular_models:dual_ehrenfried:compute_bsupp_of_jeps_dot_Sigma}
        Bsupp(j^\eps.\Sigma) = Bsupp(\Sigma) - \{ j^\eps \} \,.
    \end{align}
    Moreover, basic expansions commute, i.e.\ for two distinct basic coboundary traces $j_1^{\eps_1}$ and $j_2^{\eps_2} \in Bsupp(\Sigma)$, we have
    \begin{align}
        \label{cellular_models:dual_ehrenfried:basic_expansions_commute_formula}
        j_2^{\eps_2}.( j_1^{\eps_1}.\Sigma ) = j_1^{\eps_1}.( j_2^{\eps_2}.\Sigma ) \,.
    \end{align}
\end{prop}

\begin{proof}
    Using Lemma \ref{cellular_models:dual_ehrenfried:basic_coboundary_traces_have_unique_j_and_i}, we have
    \[
        a(j^\eps).\Sigma = (S_i\tau_h \mid \ldots \mid S_i\tau_j \mid S_{i+1}\tau_{j-1} \mid \ldots \mid S_{i+1}\tau_1) = \inhom[\tilde\tau]
    \]
    for $i = \tau_j( d_i^\Delta(a_j) )$.
    Up to an order preserving renaming of the symbols, we have
    \[
        \supp(\tilde\tau_k) \cap \supp(\tilde\tau_{k-1}, \ldots, \tilde\tau_1) = s_i^\Delta(\ \supp(\tau_k) \cap \supp(\tau_{k-1}, \ldots, \tau_1)\ )
    \]
    if $i$ is not in the support of $\tau_k$.
    Otherwise a case by case analysis yields
    \[
        \supp(\tilde\tau_k) \cap \supp(\tilde\tau_{k-1}, \ldots, \tilde\tau_1) =
            \begin{cases}
                s_{i+1}^\Delta(\ \supp(\tau_k) \cap \supp(\tau_{k-1}, \ldots, \tau_1) \ ) & \text{for } k < j \\
                s_{i}^\Delta(\ (\supp(\tau_k) \cap \supp(\tau_{k-1}, \ldots, \tau_1) \ ) - \{ i + 1\} & \text{for } k = j \\
                s_{i}^\Delta(\ \supp(\tau_k) \cap \supp(\tau_{k-1}, \ldots, \tau_1) \ ) & \text{for } k > j 
            \end{cases} \,,
    \]
    and \eqref{cellular_models:dual_ehrenfried:compute_bsupp_of_jeps_dot_Sigma} is an immediate consequence.
    
    The commutativity \eqref{cellular_models:dual_ehrenfried:basic_expansions_commute_formula} follows from
    \eqref{cellular_models:dual_ehrenfried:compute_bsupp_of_jeps_dot_Sigma} and
    the behaviour of the bijection $Bsupp(\Sigma) = Btrace(\Sigma)$ in Lemma \ref{cellular_models:dual_ehrenfried:Btrace_is_Bsupp}.
\end{proof}

\begin{defi}
    \label{cellular_models:dual_ehrenfried:expansion_of_Sigma}
    \index{cell!expansion of a cell}
    \symbolindex[j]{$J.\Sigma = j_1^{\eps_1}. \cdots. j_t^{\eps_t}.\Sigma$}{The iterated coboundary of $\Sigma$ where all $j_k$ are basic coboundary traces.}{Definition \ref{cellular_models:dual_ehrenfried:expansion_of_Sigma}}
    Consider a cell $\Sigma \in \E$ and a non-empty subset $J = \{ j_1^{\eps_1}, \ldots, j_t^{\eps_t} \} \subseteq Bsupp(\Sigma)$.
    The cell
    \[
        J.\Sigma = j_1^{\eps_1}. \cdots. j_t^{\eps_t}.\Sigma
    \]
    is called an {\bfseries expansion} of $\Sigma$.
\end{defi}

\begin{defi}
    \label{cellular_models:dual_ehrenfried:thin_cells}
    \index{cell!thin cell}
    \index{cell!set of thin cells}
    \symbolindex[t]{$\Thin$}{The set of thin cells}{Definition \ref{cellular_models:dual_ehrenfried:thin_cells}}
    A cell $\Sigma \in \E$ that is not an expansion of some other cell is called \textbf{thin}.
    The set of thin cells is $\Thin$.
\end{defi}

\begin{prop}
    \label{cellular_models:dual_ehrenfried:every_cell_is_an_expansion}
    \symbolindex[e]{$ex$}{The expansion map}{Proposition \ref{cellular_models:dual_ehrenfried:every_cell_is_an_expansion}}
    \symbolindex[p]{$\mc Pow$}{The power set operator}{Proposition \ref{cellular_models:dual_ehrenfried:every_cell_is_an_expansion}}
    Every cell $\Sigma \in \E$ is a unique expansion of a thin cell, i.e.
    denoting the power set operator by $\mc Pow$, there is a bijection
    \[
        ex \colon \coprod_{\Sigma \in \Thin} \mc Pow( Bsupp(\Sigma)) \to Cells(\E) \mspc{with}{20} Bsupp(\Sigma) \supseteq J \mapsto J.\Sigma \,.
    \]
\end{prop}

\begin{proof}
    The expansion map $ex$ is surjective by the definition of thin cells.
    
    In order to proof injectivity, consider thin cells $\tilde\Sigma$ and $\tilde\Sigma'$ together with
    $J \subseteq Bsupp(\tilde\Sigma)$ of minimal size and some $K \subseteq Bsupp(\tilde\Sigma')$ such that
    $J.\tilde\Sigma = K.\tilde\Sigma'$.
    We show that $J$ has to be empty to deduce $\tilde\Sigma = K.\tilde\Sigma'$, so $K$ is also empty (because $\tilde\Sigma$ is thin).
    
    Assume $J$ is non-empty and consider $j^\eps \in J$ and $k^\delta \in K$.
    We denote
    \[
        \Sigma = (J - \{j^\eps\}).\tilde\Sigma = \inhom \mspc{and}{20} \Sigma' = (K - \{k^\delta\}).\tilde\Sigma' = \inhom[\tau'] \,.
    \]
    By assumption,
    \begin{align}
        j^\eps.\Sigma
            &= (S_a\tau_h \mid \ldots \mid S_a\tau_j \mid S_{a+1}\tau_{j-1} \mid \ldots \mid S_{a+1}\tau_1) \\
            \label{cellular_models:dual_ehrenfried:every_cell_is_an_expansion_comparision_of_two_sequences}
            &= (S_b\tau_h' \mid \ldots \mid S_b\tau_k' \mid S_{b+1}\tau_{k-1}' \mid \ldots \mid S_{b+1}\tau_1') = k^\delta\Sigma'
    \end{align}
    for some unique $a$ and $b$.
    
    Here, $a=b$ is impossible:
    If $j = k$, the index set $J$ was clearly not minimal,
    but for $j > k$, we have $S_{a+1}\tau_k = S_a\tau_k'$ with $a \in \supp(\tau_k)$ by (ii) in Definition \ref{cellular_models:dual_ehrenfried:basic_coboundary_traces},
    so $\supp(S_{a+1}\tau_k) \not\ni a+1 \in \supp(S_a\tau_k')$.
    
    Without loss of generality, let $a < b$.
    Similar to the previous consideration, $a+1 = b$ and $j > k$ must not hold at once as otherwise $S_a\tau_j = S_b\tau_j' = S_{a+1}\tau_j'$ with $\supp(S_{a+1}\tau_j) \ni a+1 \not\in \supp(S_{a+1}\tau_j')$.
    
    Now that we excluded all troublesome cases, the transpositions of $j^\eps.\Sigma$ and $k^\delta.\Sigma'$ at the $l\Th$ spot are
    \[
        S_c\tau_l = S_d\tau_l'
    \]
    for appropriate $c < d$.
    We deduce
    \[
        S_d\tau_l' = S_cS_{d-1}\tau'' \mspc{with}{20} \tau_l'' = D_c\tau_l' \neq 1_\SymGr
    \]
    as (by the identities in Proposition \ref{notation:DiSj_identities})
    \[
        S_c\tau_l = S_cD_cS_c\tau_l \mspc{and}{20} D_cS_d\tau_l'' = S_{d-1}D_c\tau_l' \,.
    \]
    Substituting the transpositions of $k^\delta.\Sigma'$ in equation \eqref{cellular_models:dual_ehrenfried:every_cell_is_an_expansion_comparision_of_two_sequences},
    it is readily seen that $J$ was not minimal.
\end{proof}
