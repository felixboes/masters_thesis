\documentclass[12pt]{beamer}            % Latex-Beamer Klasse. Um ein Handout zu produzieren reicht es aus, die Option von [12pt] zu [12pt, handout] zu ändern.
\usefonttheme[onlymath]{serif}          % "Normale" Mathemathik-Schriftart in Latexbeamer verwenden.
\setbeamertemplate{footline}[frame number]    % Seitenzahl unten anzeigen.
\setbeamertemplate{navigation symbols}{}      % Die komische Navigationsleiste ausschalten.

% Pakete
\usepackage[linesnumbered,ruled,vlined]{algorithm2e}    % Algorithmen setzen.
\usepackage{amsmath,amssymb,amsthm,amsfonts,amsbsy,latexsym}    % "Notwendige" AMS-Math Pakete.
\usepackage{array}                      % Bessere Tabellen.
\renewcommand{\arraystretch}{1.15}      % Tabellen bekommen ein wenig mehr Platz.
\usepackage{bbm}                        % Dicke 1.
\usepackage[hypcap]{caption}            % Damit Hyperrefs bei der figure-Umgebung auf die Figure zeigt statt auf die Caption.
\usepackage{diagbox}                    % Diagonale in Tabellen.
\usepackage{enumitem}                   % Zum Ändern der Nummerierungsumgenung 'enumerate'
\setlist[enumerate,1]{label=(\roman*)}  % Aufzählungen sind vom Typ 'Klammer auf; kleine römische Zahl; Klammer zu'
\usepackage[T1]{fontenc}                % Bessere Schrift
\usepackage{ifthen}                     % Zum checken ob Parameter leer sind.
\usepackage[utf8]{inputenc}             % utf8 als Eingabeformat.
\usepackage{lmodern}                    % Bessere Schrift
\usepackage{mathtools}                  % Subscript unter Summen behandeln. Der Befehl lautet \mathclap.
\usepackage{multirow}                   % In Tabellen mehrere Zeilen zu einer machen.
\usepackage{rotating}                   % Um Figures zu drehen.

\graphicspath{{pictures/}}              % Pfad in dem die mit Inkscape erstellen Bilder liegen (relativ zum Hauptverzeichnis).

% Titel
\author{Anna Hermann}
\title{Abschlussvortrag - Teil 1}
\subtitle{}
\date{12.09.2015}

\begin{document}

% \frame{\titlepage} % Gibt die Titelseite aus

\begin{frame}
    \begin{figure}[ht]
    \centering
    \def\svgwidth{0.9\columnwidth}
    \input{pictures/intro_trousers.pdf_tex}
  \end{figure}
\end{frame}

\begin{frame}
    \begin{figure}[ht]
    \centering
    \def\svgwidth{0.9\columnwidth}
    \input{pictures/intro_trousers_flat.pdf_tex}
  \end{figure}
\end{frame}

\begin{frame}
    \begin{figure}[ht]
    \centering
    \def\svgwidth{0.9\columnwidth}
    \input{pictures/intro_trousers_radial.pdf_tex}
  \end{figure}
\end{frame}

\begin{frame}
\begin{figure}[ht]
    \centering
    \def\svgwidth{.99\columnwidth}
    {\input{pictures/flow_with_one_puncture_with_equipotential_lines.pdf_tex}}
    \hfill
    \def\svgwidth{.33\columnwidth}
    {\input{pictures/flow_with_one_puncture_slit_picture.pdf_tex}}
    \end{figure}
\end{frame}

\begin{frame}
    \begin{figure}[ht]
    \centering
    \def\svgwidth{0.7\columnwidth}
    \input{pictures/pole_with_order_3.pdf_tex}
  \end{figure}
\end{frame}

\begin{frame}
 \begin{center}
  \begin{tabular}{|c|c|c|l||r|r|}
      \cline{1-6}
      \multicolumn{6}{|c|}{Runtime and Memory Results for $g = 3$, $m = 1$} \\ \hline \hline
      Radial & CSS & Bool & Threads &  \multicolumn{1}{c}{Runtime} & \multicolumn{1}{|c|}{Memory} \\ 
             &     &      & \multicolumn{1}{c||}{\#} & \multicolumn{1}{c}{[h : min : s]} & \multicolumn{1}{|c|}{[MB]} \\ \hline\hline
      x      & x   & x    & $t = 12$  & $0:27:43$           & $7056$  \\ \hline
      x      & x   & x    & $t = 1$               & $1:37:49$           & $7031$  \\ \hline
      x      &     & x    & $t = 12$  & $1:25:39$           & $10819$ \\ \hline
      x      & x   &      & $t = 12$  & $18:36:22$          & $92857$ \\ \hline
             & x   & x    & $t = 12$  & /                   & /       \\ \hline
  \end{tabular}
\end{center}
\end{frame}

\end{document}