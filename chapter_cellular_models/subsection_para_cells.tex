\subsection{Cells in Homogeneous Notation}
\label{cellular_models:parallel:cells_in_homogenous_notation}
In this subsection, the partition $r = r_1 + \ldots + r_n$ is meant to be fixed.
We define arbitrary cells $\Sigma$ of bidegree $(p,q)$ in the homogenous notation.
Before going into details, recall our notation for the symmetric groups (see Appendix \ref{notation:semi_simpl_sym_grp}).

\begin{defi}
    \label{cellular_models:parallel:partition}
    \index{partition into levels}
    \symbolindex[p]{$[p] = \{ \ul 0_1, \ul 1_1, \ldots, \ul p_1, \ldots, \ul 0_r, \ul 1_r, \ldots, \ul p_r \}$}
        {A partition of $p$ into $r$ levels.}
        {Definition \ref{cellular_models:parallel:partition}}
    Consider an ordered partition of the natural number $p = p_1+\ldots+p_r$ with all $p_i$ positive.
    The set
    \[
        [p] = \{ \ul 0_1, \ul 1_1, \ldots, \ul{p_1}_1, \ldots, \ul 0_r, \ul 1_r, \ldots, \ul{p_r}_r \}
    \]
    consisting of $p+r$ elements is a {\bf partition of $p$ into $r$ levels}.
    In what follows, we will abuse notation by abbriviating
    \[
        \ul p_k = \ul{p_k}_k \,.
    \]
    This should not cause any confusion.
    The set $[p]$ is ordered canonically via
    \[
        \ul0_1 < \ul1_1 < \ldots < \ul p_1 < \ldots < \ul 0_r < \ul 1_r < \ldots < \ul p_r \,.
    \]
\end{defi}

\begin{defi}
    \label{cellular_models:parallel:homogeneous_notation}
    \index{cell!parallel cell in homogeneous notation}
    \symbolindex[s]{$\Sigma = (\sigma_q, \ldots, \sigma_0)$}{A parallel cell written in homogeneous notation}{Definition \ref{cellular_models:parallel:homogeneous_notation}}
    Using the {\bf homogeneous notation}, a combinatorial cell of bidegree $(p,q)$ with respect to a given partition $[p]$ is a $(q+1)$-tuple of permutations $\sigma_j \in \Symgrp_{[p]}$
    \[
        \Sigma = \homogq \,.
    \]
    Most of the time we refere to $\Sigma$ as a cell on $r$ levels, leaving the partition $[p]$ unmentioned.
\end{defi}

\begin{defi}
    \label{cellular_models:parallel:inner_cells}
    \index{cell!inner parallel cell}
    \index{levels!ascendingly ordered levels}
    A cell $\Sigma = \homogq$ of bidegree $(p,q)$ is called (parallel) {\bf inner cell} if it is subject to the following conditions.
    \begin{enumerate}
        \item Every $\sigma_i$ maps $\ul p_k$ to $\ul 0_k$ for every $k$.
        \item The zero${}^{\text{th}}$ permutation $\sigma_0$ is fixed to be $\sigma_0 = (\ul 0_1\ \ul 1_1\ \ldots\ \ul p_1)\ldots (\ul 0_r\ \ul 1_r\ \ldots\ \ul p_r)$.
        \item For every $0 \le i < q$, the permutations $\sigma_i$ and $\sigma_{i+1}$ are distinct.
        \item There is no symbol $\ul 0_k \le \ul j_k < \ul p_k$ that is mapped to its successor $\ul{j+1}_k$ by all permutations $\sigma_i$.
        \item The levels of $\Sigma$ are {\bf ordered ascendingly} with respect to the boundary curves,
            i.e.\ $\sigma_q$ induces on $\{\ul0_1, \ldots, \ul0_r\} \subseteq [p]$ the permutation
            $(\ul0_1\ \ldots\ \ul0_{p_1})(\ul0_{p_1 + 1} \ \ldots\ \ul0_{p_1+p_2}) \ldots (\ul0_{p_{r-1} + 1} \ \ldots\ \ul0_{p_r})$.
    \end{enumerate}
\end{defi}

Reversing the process discussed in Section \ref{cellular_models:from_moduli_spaces_to_parallel_slit_domains}, an inner cell on one level is pictured as follows.
We start off with the unit square which is cut up into horizontal stripes of the same size.
The stripes are denoted by the symbols $\ul0$ to $\ul p$.
Each stripe is divided into rectangles, denoted by the numbers $0$ to $q$,
hence we subdivided the initial square into $(p+1)(q+1)$ pieces of the same size.
Now we glue the stripes according to the permutations $\homogq$ as indicated by Figure \ref{cellular_models:parallel:homogeneous_glueing}.
\begin{figure}[ht]
\centering
\incgfx{pictures/cellular_para_homogeneous_glueing}
\caption{\label{cellular_models:parallel:homogeneous_glueing}Glueing a surface from a cell in homogeneous notation.
    The thick line indicates the boundary curve (which is seen as a parametrized disc around a pole).}
\end{figure}
The top side of the rectangle with coordinate $(j,\ul i)$ is glued to the bottom side of the rectangle with coordinate $(j, \sigma_j(\ul i))$, but we omit to glue the $\ul p\Th$ rectangle to $\ul0\Th$.
After glueing all pieces, we receive a surface with punctures and a boundary curve (which we understand as the boundary of a contractible neighbourhood of the dipole) as follows.
The cycle of $\sigma_q$ containing $\ul0$ corresponds to the boundary curve (and therefore to the dipole), which we indicate by a thick line in Figure \ref{cellular_models:parallel:homogeneous_glueing}.
All the other cycles of $\sigma_q$ resemble the punctures of the surface.

The picture for an inner cell on $r$ levels is similar.
Here we start off with $r$ unit squares $A_1, \ldots, A_r$, cut each $A_k$ in $(p_k+1)(q+1)$ pieces and glue the collection of all pieces according to the permutations $\sigma_i$.
Observe that the resulting surface has exactly $n$ boundary curves by it may be disconnected.
We will treat this case in the next definition.
The cell in Figure \ref{cellular_models:parallel:closed_disc} resembles a closed disc and corresponds to the example in the previous section, where we cut $\mathbb S^2$ into two pieces along the critical flow of the dipole function $\Re(z^2)$.
\begin{figure}[ht]
\centering
\incgfx{pictures/cellular_para_closed_disc}
\caption{\label{cellular_models:parallel:closed_disc}The cells $\big((\ul 0_1\ \ul 1_2\ \ul 0_2\ \ul 1_1): (\ul 0_1\ \ul 1_1)(\ul 0_2\ \ul 1_2)\big)$ resembles a closed disc and is understood as the complement of an open disc around infinity in $\mathbb S^2$.}
\end{figure}

\begin{defi}
    \label{cellular_models:parallel:connected}
    \index{cell!connected cell}
    A combinatorial cell is {\bf connected} if the resulting surface is connected.
    Wandering on the surface by traversing through the stripes horizontally or (using the glueing information) vertically, we conclude that a cell is connected if and only if the equivalence relation on $[p]$ generated by
    \[
        i \sim j \iff \exists k : j = \sigma_k(i)
    \]
    consists of exactly one element.
\end{defi}

\subsection{Cells in Inhomogeneous Notation[B]}
\label{cellular_models:parallel:cells_in_inhomogenous_notation}
As before, the ordered partition $r = r_1 + \ldots + r_n$ is meant to be fixed.
It is fertile to give an equivalent description for inner cells of bidegree $(p,q)$.
Let $\Sigma = \homogq$ be a inner cell and consider the permutations $\tau_j = \sigma_j \sigma_{j-1}^{-1}$ for $1 \le j \le h$.
Every symbol $\ul 0_k$ is fixed by every $\tau_i$.
We sometimes view these as permutations on the symbols $[p] - \{ \ul 0_k \mid 1 \le k \le r\}$.
In the next definition we rephrase the conditions of Definition \ref{cellular_models:parallel:inner_cells}.

\begin{defi}
    \label{cellular_models:parallel:inhomogeneous_notation}
    \index{cell!parallel cell in inhomogeneous notation}
    \symbolindex[s]{$\Sigma = (\tau_q \mid \ldots \mid \tau_1)$}{A parallel cell written in inhomogeneous notation}{Definition \ref{cellular_models:parallel:inhomogeneous_notation}}
    Using the {\bf inhomogeneous notation}, a combinatorial cell of bidegree $(p,q)$ with respect to a given partition $[p]$
    is a $q$-tuple of permutations $\tau_j \in \Symgrp_{[p]}$ written as
    \[
        \Sigma = \inhomq \,.
    \]
    It is a (parallel) {\bf inner cell} if it is subject to the following conditions
    \begin{enumerate}
        \item every permutation $\tau_q, \ldots, \tau_1$ is non-trivial,
        \item the set of common fixed points of the permutations $\tau_q, \ldots, \tau_1$ is exactly $\{ \ul0_1, \ldots, \ul0_r\}$ and
        \item the permutation $\tau_q \cdots \tau_1 \sigma_0 \in \Symgrp_{[p]}$ induces on $\{\ul0_1, \ldots, \ul0_r\} \subseteq [p]$ the permutation
            $(\ul0_1\ \ldots\ \ul0_{p_1})(\ul0_{p_1 + 1} \ \ldots\ \ul0_{p_1+p_2}) \ldots (\ul0_{p_{r-1} + 1} \ \ldots\ \ul0_{p_r})$.
    \end{enumerate}
\end{defi}
The following is clearly a one-to-one correspondence of inner cells with respect to a given partition $[p]$
\[
    \homogq \mapsto \inhomq \mspc{with}{20} \tau_i = \sigma_i\sigma_{i-1}^{-1} \,.
\]

In contrast to the homogeneous notation, where the combinatorial information describes how to traverse through the geometric cell vertically,
the inhomogeneous notation portrays the tours around each (inner) corner point, which is a stagnation point in the light of Section \ref{cellular_models:from_moduli_spaces_to_parallel_slit_domains}.
In Figure \ref{cellular_models:parallel:cell_comparison_notations} we picture this for the cell $\Sigma = \big( (\ul0\ \ul2): (\ul0\ \ul1\ \ul2) \big)$ (written in homogeneous notation),
which should remind the reader of the example portrayed in Figure \ref{cellular_models:from_moduli_spaces_to_parallel_slit_domains:flow_with_one_puncture_with_equipotential_lines}.
\begin{figure}[ht]
\centering
\incgfx{pictures/cellular_para_cell_comparison_notations}
\caption{\label{cellular_models:parallel:cell_comparison_notations}Comparison of the homogeneous and inhomogeneous notation.}
\end{figure}

Both the number of punctures and the number of boundary components of the corresponding surface are encoded by the permutation $\sigma_q$.
\begin{defi}
    \label{cellular_models:parallel:number_cycles}
    \label{cellular_models:parallel:number_punctures}
    \label{cellular_models:parallel:number_boundaries}
    \label{cellular_models:parallel:norm}
    \index{cell!number of cycles of a parallel cell}
    \index{cell!number of punctures of a parallel cell}
    \index{cell!number of boundaries of a parallel cell}
    \index{cell!norm of a parallal cell}
    \symbolindex[n]{$\ncyc(\Sigma)$}{The number of cycles of a parallel cell $\Sigma$.}{Definition \ref{cellular_models:parallel:number_cycles}}
    \symbolindex[m]{$m(\Sigma)$}{The number of punctures of a parallel cell $\Sigma$.}{Definition \ref{cellular_models:parallel:number_punctures}}
    \symbolindex[n]{$n(\Sigma)$}{The number of boundaries of a parallel cell $\Sigma$.}{Definition \ref{cellular_models:parallel:number_boundaries}}
    \symbolindex[n]{$N(\Sigma)$}{The norm of a parallel cell $\Sigma$.}{Definition \ref{cellular_models:parallel:norm}}
    Consider an arbitrary cell $\Sigma = \homogq$ of bidegree $(p,q)$.
    \begin{enumerate}
        \item The {\bf number of cycles} of $\Sigma$ is defined to be the number of cycles of the permutation $\sigma_q$
            \[
                \ncyc( \Sigma ) = \ncyc( \sigma_q) \,,
            \]
            where we view fixed points as cycle of length zero.
        \item Every cycle of $\sigma_q$ that contains at least one symbol $\ul 0_k$ is called {\bf boundary cycle} of $\Sigma$.
            The {\bf number of boundaries} of $\Sigma$ is denoted by
            \[
                n(\Sigma) = \#\{ \text{boundary cycle of } \Sigma\}
            \]
            and the {\bf number of punctures} of $\Sigma$ is
            \[
                m(\Sigma) = \ncyc(\Sigma) - n(\Sigma) \,.
            \]
        \item The {\bf norm} of $\Sigma$ is
            \[
                N(\Sigma) = N(\sigma_q\sigma_{q-1}^{-1}) + \ldots + N(\sigma_1\sigma_0^{-1}) \,,
            \]
            where $N$ measures the word length in the symmetric group $\Symgrp_{[p]}$ with respect to the set of all transpositions.
    \end{enumerate}
\end{defi}

\begin{rem}
    \label{cellular_models:parallel:h_m_n_g_are_recognized}
    \index{cell!genus of a parallel cell}
    \symbolindex[g]{$g(\Sigma)$}{The genus of a parallel cell $\Sigma$.}{Remark \ref{cellular_models:parallel:h_m_n_g_are_recognized}}
    Reversing the dissection process in Section \ref{cellular_models:from_moduli_spaces_to_parallel_slit_domains},
    the number of punctures and boundary curves of a combinatorial cell $\Sigma$
    equals the number of punctures and boundary curves of the surface $F$ that is obtained by glueing.
    Moreover, an inner cell $\Sigma = \inhomq$ defines an imbedded connected graph $\mc K_0$ whose complement are $r$ basins.
    The poles, punctures and stagnation points correspond to the vertices and $N(\tau_j)+s$ is the number of edges that end in the $s$ stagnations at the $j\Th$ equipotential line.
    The Euler characteristic of $F$ is
    \[
        2-2g = \chi(F) = \#vertices - \#edges + \#faces = m+n - h + r\,,
    \]
    so the genus of $F$ is uniquely determined by $h,m$ and the partition.
    \[
        g(\Sigma) = \frac{h - m - n - r + 2}{2}
    \]

\end{rem}
