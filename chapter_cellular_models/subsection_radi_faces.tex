\subsection{Faces}
\label{cellular_models:radial:faces}
\index{cell!vertical face of a radial cell}
\symbolindex[d]{$d'_j(\Sigma)$}{The $j\Th$ vertical face of a radial cell $\Sigma$}{Subsection \ref{cellular_models:radial:faces}}
\index{cell!horizontal face of a radial cell}
\symbolindex[d]{$d''_i(\Sigma)$}{The $i\Th$ horizontal face of a radial cell $\Sigma$}{Subsection \ref{cellular_models:radial:faces}}

Using the same formulas as in parallel case (see Definitions \ref{cellular_models:parallel:dv} and \ref{cellular_models:parallel:dh}), 
we define \textbf{vertical} and \textbf{horizontal faces} for radial cells $\Sigma$ of bidegree $(p, q)$.
In particular, note that Proposition \ref{cellular_models:parallel:prop_dh} also holds for radial inner cells.
Geometrically, the $j\Th$ vertical face of $\Sigma$ arises from $\Sigma$ by deleting its $j\Th$ concentrical stripe, 
for $j \in \{0, \dotsc, q\}$.
The $\ul i\Th$ horizontal face arises by deleting the $\ul i\Th$ radial segment for $\ul i \in [p]$ (see Figure \ref{cellular_models:radial:comparison_face_operators}).
\begin{figure}[ht]
\centering
\incgfx{pictures/cellular_radi_comparison_face_operators}
\caption{\label{cellular_models:radial:comparison_face_operators}The vertical and horizontal face operators.}
\end{figure}
